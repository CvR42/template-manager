\chapter{The text substitution language}
\label{s.tm}
The standard modules for the various languages are implemented
with the text substitution language described in this section.
\par
One should only use the text substitution language directly if the available
modules do not meet the requirements,
since writing in this language is not exactly a pleasure.
The casual user of {\Tm} is expected to use only the standard modules that
are included in the {\Tm} package.
For this purpose it is sufficient to know
the {\tt .include} command (to invoke the modules),
the {\tt .insert} command (to insert configuration files),
the {\tt .set} command (to set some parameters for the modules),
and the {\tt ..} command (to add comments to the {\Tm} input).
\section{Conventions}
Boolean functions return the string `{\tt 0}' for
{\it false} \/or the string `{\tt 1}' for {\it true}.
At places where a boolean expression is expected,
the string `{\tt 0}' is {\it false},
all other strings are {\it true}.
\par
A \dfn{number} is a string of digits with an optional sign before it.
\par
A word is either a string of non-blank characters, or a string of
arbitrary characters surrounded by double quotes. A quoted string
must end before the end of the line, and it may not contain
double quote characters.
Thus,
the following lines each contain one word:
\begin{verbatim}
a
word
"this is quoted and therefore treated as one word"
\end{verbatim}
\section{Line commands}
Each time a dot (`{\tt .}') is encountered as the first character of a line,
that line is interpreted as a
{\tt {\Tm}} command.
If the second character of the line is also a dot,
the line is comment, and is completely ignored.
Otherwise the first word after the dot
(possibly preceded by some spaces and tabs)
is interpreted as the line command,
and the remainder of the line is interpreted as a list of parameters
that is interpreted each time the command is encountered.
Evaluation of the parameters does {\em not} \/occur for termination commands
({\tt .endif}, {\tt .endforeach}, {\tt .endwhile}).
\par
The following line commands are supported:
\begin{verbatim}
.append <var> <val>..<val>
\end{verbatim}
\begin{desc}
\index{.append@\verb+.append+}
Append the given list of values after the current values of variable
{\tt <var>}.
\end{desc}
\begin{verbatim}
.call <macro> <parm>..<parm>
\end{verbatim}
\begin{desc}
Invoke macro \verb+<macro>+ with the given list of parameters.
Before execution of the macro, the formal parameters of the macro
are set to the values in the parameter list.
The number of parameters must match the number of formal parameters,
or else an error message is given.
To pass a list of words to a formal parameter,
surround it with double quotes.
Macros `see' the variable values and macros that are in effect at the moment
of execution of the macro.
All changes that are made to variables or macros within a macro
are invisible outside the macro.
\par
During execution of the macro \verb+.return+ commands may be given
to record a return value for the macro, but in the \verb+.call+ command
this value is ignored.
If you want to use it,
use the \verb+${call <macro> <parm>..<parm>}+ form instead.
\end{desc}
\begin{verbatim}
.error <line>
\end{verbatim}
\begin{desc}
\index{.error@\verb+.error+}
Write {\tt <line>} to the error output stream.
This is useful for debugging or error messages.
\end{desc}
\begin{verbatim}
.exit <n>
\end{verbatim}
\begin{desc}
\index{.exit@\verb+.exit+}
Stop {\Tm} immediately with exit code {\tt <n>}.
This is usually used to halt upon fatal errors in a template.
\end{desc}
\begin{verbatim}
.expandinherits <nm>..<nm>
\end{verbatim}
\begin{desc}
\label{s.expandinherits}
\index{.expandinherits@\verb+.expandinherits+}\index{inherits!expanding}
Given a list of types, replace the internal datastructure description of
each type with a new version were inherit references have been expanded.
\par
In the expanded version of a datastructure, all references to
inherited types have been removed, and all of the fields of the
directly or indirectly inherited types have been inserted in front of
the existing fields of the tuple or constructors.
This expansion is only possible for inheritance from tuple types, not
from constructor or primitive types.
\par
Since the inherit references are removed,
subsequent invocations of \verb'.expandinherits' for the same
type do not alter the data structure descriptions any more.
\par
For example, the data structure definitions
\begin{verbatim}
serial == ( serialno:int );

tag == serial + ( name:tmstring );

expr ::= tag +
    ExprPlus a:expr b:expr |
    ExprNeg v:expr |
    ExprConst n:int;
\end{verbatim}
are expanded by \verb'.expandinherits tag expr' to:
\begin{verbatim}
serial == ( serialno:int );

tag == ( serialno, name:tmstring );

expr ::=
    ExprPlus serialno:int name:tmstring a:expr b:expr |
    ExprNeg serialno:int name:tmstring v:expr |
    ExprConst serialno:int name:tmstring n:int;
\end{verbatim}
\end{desc}
\begin{verbatim}
.foreach <var> <val>..<val>
\end{verbatim}
\begin{desc}
\index{.foreach@\verb+.foreach+}
Set variable {\tt <var>} to each of the values {\tt <val>} and do a
repeated translation of all following lines up to the next unbalanced
{\tt .endforeach}
for each of these values of {\tt <var>}.
\end{desc}
\begin{verbatim}
.if <expr>
\end{verbatim}
\begin{desc}
\index{.if@\verb+.if+}
Evaluate {\tt <expr>}.
If {\tt <expr>} is {\it true},
translate the following lines up to the next unbalanced
{\tt .else} or {\tt .endif}
and if relevant skip all lines between {\tt .else} and {\tt .endif}.

If {\tt <expr>} is {\it false},
only translate all lines between the following unbalanced
{\tt .else} and {\tt .endif}.
If no {\tt .else} is encountered, no translation is done.
\end{desc}
\begin{verbatim}
.include <filename>
\end{verbatim}
\begin{desc}
\index{.include@\verb+.include+}
Use the {\Tm} commands in {\tt <filename>} as if they were inserted at
this position.
It is not allowed to leave commands unterminated in {\tt <filename>}.
Nested {\tt .include} or {\tt .insert} commands can occur to an arbitrary depth.
No checking is done on recursion!
All changes to variables that are done in this module are invisible after
the {\tt .include} has terminated.
\par
All changes to variables that are done in the {\tt .include}-ed file
are discarded after translation of the file. Thus, as far as the invoking
file is concerned, an {\tt .include}-ed file does not change any variables.
This is useful for the invocation of standard translation modules.
\par
If the file cannot be opened, {\Tm} searches a list of directories for it.
By default this list contains the directory of the variable
\verb+libpath+, but the user may supply additional directories with the
\verb+-I+ command line option.
\end{desc}
\begin{verbatim}
.insert <filename>
\end{verbatim}
\begin{desc}
\index{.insert@\verb+.insert+}
Use the {\Tm} commands in {\tt <filename>} as if they were inserted at
this position.
It is not allowed to leave commands unterminated in {\tt <filename>}.
Nested {\tt .include} or {\tt .insert}
commands can occur to an arbitrary depth.
No checking is done on recursion!
\par
The variable settings that are done in an {\tt .insert}
file remain visible outside the file.
This is useful for configuration files or other files that
must make global changes to variables.
\par
If the file cannot be opened, {\Tm} searches a list of directories for it.
By default this list contains the directory of the variable
\verb+libpath+, but the user may supply additional directories with the
\verb+-I+ command line option.
\end{desc}
\begin{verbatim}
.macro <name> <parm>..<parm>
\end{verbatim}
\begin{desc}
Define a new macro \verb+<name>+ with the given list of formal parameters.
Any previous macro of this name is replaced by the new definition.
The body of the macro consists of the lines after the
\verb+.macro+ up to the next unbalanced \verb+.endmacro+.
A macro may contain any {\Tm} command, including other \verb+.macro+
definitions.
Macros may return a list of words (see the \verb+.return+ command),
and may be invoked with the line command \verb+.call+ or an expression
of the form \verb+${call <macro> parm..parm}+.
See the description of these commands for further details.
\end{desc}
\begin{verbatim}
.redirect <filename>
\end{verbatim}
\begin{desc}
\index{.redirect@\verb+.redirect+}
Redirect the output up to the balancing \verb'.endredirect' to the given
file.
The parameter list of the \verb'.redirect' command must evaluate to
exactly one filename.
\end{desc}
\begin{verbatim}
.return <val>..<val>
\end{verbatim}
\begin{desc}
Record the given values as return value of the current macro.
Any previous return value is replaced by this one.
Contrary to the {\tt return} command in {\C}, this \verb+.return+
command does {\em not} \/terminate execution of the macro.
\end{desc}
\begin{verbatim}
.set <var> <val>..<val>
\end{verbatim}
\begin{desc}
\index{.set@\verb+.set+}
Set variable {\tt <var>} to the given list of values.
\end{desc}
\begin{verbatim}
.while <expr>
\end{verbatim}
\begin{desc}
\index{.while@\verb+.while+}
Evaluate {\tt <expr>} and translate the subsequent lines up to the next
unbalanced {\tt .endwhile} if {\tt <expr>} is {\it true}.
Repeat this until {\tt <expr>} is {\it false}.
\end{desc}
For example:
\begin{verbatim}
.set l 1
.foreach n 2 3 4 5
.append l $n
.endforeach
$l
\end{verbatim}
(where \verb+$n+ denotes a reference to variable \verb+n+) will produce:
\begin{verbatim}
1 2 3 4 5
\end{verbatim}
And an other example (the meaning of \verb+${}+ and \verb+$[]+ expressions
will be explained in sections~\ref{s.fn} and~\ref{s.intexpr}):
\begin{verbatim}
.. Example macro: calculate !n
.macro fac n
.if $[$n<2]
.return 1
.else
.return $[$n*${call fac $[$n-1]}]
.endif
.endmacro
.. Now test it out:
.foreach n 1 2 4 5 7
$n! = ${call fac $n}
.endforeach
\end{verbatim}
Will result in:
\begin{verbatim}
1! = 1
2! = 2
4! = 24
5! = 120
7! = 5040
\end{verbatim}
\section{Variables}
Variables are referenced with the expression
{\tt \$(<varname>)} or optionally {\tt \$<letter>} for single character
variables.
Each time the characters `{\tt \$(}' are encountered,
the evaluated string up to the next unbalanced `{\tt )}' 
is interpreted as the name of a variable.
If `{\tt \$<letter>}' or `{\tt \$<digit>}' is encountered it is interpreted as
a reference to a variable with name `{\tt <letter>}' or {\tt <digit>}
respectively.
\par
For example:
\begin{verbatim}
.set b +
.set bla b
$b, $(b), $bla, $(bla), $($(bla))
\end{verbatim}
will produce:
\begin{verbatim}
+, +, +la, b, +
\end{verbatim}
Note that the result for `\verb+$bla+' is `{\tt +la}', since the
variable `{\tt b}' is referenced, not the variable `{\tt bla}'.
\par
If the characters `{\tt \$\{}' are encountered the evaluated string up to
the next unbalanced `{\tt \}}' is interpreted as a function invocation,
and the result of that function replaces the invocation string,
see section~\ref{s.fn}.
If the characters `{\tt \$[}' are encountered the evaluated string up
to the next unbalanced `{\tt ]}' is interpreted as an integer expression,
and the result of the evaluation of that expression replaces the expression,
see section~\ref{s.intexpr}.
If any other character is encountered after the `{\tt \$}',
it is copied literally.
\par
The following variables are predefined:
\par
\begin{desctab}
{\tt libpath}\index{libpath@{\tt libpath}}
&
The absolute directory path of the library of standard modules.
This may differ between installations,
so use this variable to locate the library, or let {\Tm} locate
it using its search path.
By default Tm uses a built-in path, but you can change this by setting
the environment variable TMLIBPATH\index{TMLIBPATH@\verb'TMLIBPATH'}.
\\
{\tt pathsep}\index{pathsep@{\tt pathsep}}
&
The character used as separation character between elements in a directory
path. This is operating system dependent.
\\
{\tt kernel-version}\index{kernel-version@{\tt kernel-version}}\index{version!Tm kernel version number}
\index{Tm kernel!version number}
&
The version number of the {\Tm} kernel, of which {\Tm} itself is a part.
\\
{\tt tmvers}\index{tmvers@{\tt tmvers}}\index{version!Tm version number}
\index{Tm!version number}
&
The version number of {\Tm} itself.
Not that this version number is different from the version number of the
kernel, of which {\Tm} is a part.
{\tt tmvers} is an integer, and is used in templates to determine whether
the version of {\Tm} that is executing the template is recent enough.
{\tt tmkernel-version} is a string, and is used to label the generated
code.
\\
{\tt verbose}\index{verbose@{\tt verbose}}
&
The verbose flag.
It is {\it true} \/if the flag {\tt -v} has been given at the command line,
else {\it false}.
\end{desctab}
\par
The variables {\tt listpre} and {\tt listsuff} are used by the functions
\verb+mklist+, \verb+stemname+ and \verb+deptype+ to construct list type
names from elements names or vice versa.
\section{Functions}
\label{s.fn}
Functions are invoked with the expression
\begin{verbatim}
${<fnname> <par>..<par>}
\end{verbatim}
Each time the characters `{\tt \$\{}' are encountered,
the string up to the next unbalanced `{\tt \}}' is evaluated,
and the resulting string is
interpreted as a function invocation consisting of a function with the name
{\tt <fnname>} possibly followed by a number of parameter words.
\subsection{Comparison functions}
Most comparison functions only work on numbers---and will complain if they
get anything else than a number---but {\tt eq} and {\tt neq} work on strings.
Do {\it not} \/compare numbers with these functions
since this is not safe (to these functions `{\tt 0}'
is different from `{\tt 00}').
\par
\begin{desctab}
{\tt eq sa sb}\index{eq@{\tt eq}}
&
Return {\it true} \/if the string $\verb'sa' = \verb'sb'$.
Strings are equal if they are of the same length and are have the
same character at the same position in the string.
To compare numbers use {\tt ==}.
\\
{\tt neq sa sb}\index{neq@{\tt neq}}
&
Return {\it true} \/if the string ${\tt sa} \not= {\tt sb}$.
To compare numbers use \verb'!='.
\\
{\tt == na nb}\index{==@{\tt ==}}
&
Return {\it true} \/if ${\tt na} = {\tt nb}$.
To compare strings use {\tt eq}.
\\
\verb '!= na nb'\index{"!=@{\tt "!=}}
&
Return {\it true} \/if ${\tt na} \not= {\tt nb}$.
To compare strings use {\tt neq}.
\\
{\tt < na nb}\index{<@{\tt <}}
&
Return {\it true} \/if ${\tt na} < {\tt nb}$.
\\
{\tt <= na nb}\index{<=@{\tt <=}}
&
Return {\it true} \/if ${\tt na} \leq {\tt nb}$.
\\
{\tt > na nb}\index{>@{\tt >}}
&
Return {\it true} \/if ${\tt na} > {\tt nb}$.
\\
{\tt >= na nb}\index{>=@{\tt >=}}
&
Return {\it true} \/if ${\tt na} \geq {\tt nb}$.
\end{desctab}
For example:
\begin{verbatim}
${eq bla bla}, ${eq 00 0}, ${== 00 0}, ${< 1 2}, ${< 2 1}
\end{verbatim}
will produce:
\begin{verbatim}
1, 0, 1, 1, 0
\end{verbatim}
\subsection{Arithmetic functions}
All arithmetic functions only work on numbers.
\par
\nopagebreak
\begin{desctab}
{\tt + n1..nn}\index{+@{\tt +}}
&
Return ${\tt n1} + \cdots + {\tt nn}$.
The sum of an empty list is 0.
\\
{\tt - na nb}\index{-@{\tt -}}
&
Return ${\tt na} - {\tt nb}$.
\\
{\tt * n1..nn}\index{*@{\tt *}}
&
Return ${\tt n1} \times \cdots \times {\tt nn}$.
The product of an empty list is 1.
\\
{\tt / na nb}\index{/@{\tt /}}
&
Return ${\tt na} / {\tt nb}$.
\\
{\tt \% na nb}\index{"%@\verb+%+}
&
Return the remainder of the division of {\tt na} by {\tt nb}.
\\
{\tt max n1..nn}\index{max@{\tt max}}
&
Return the maximum of ${\tt n1} \cdots {\tt nn}$.
The list must contain at least 1 element.
\\
{\tt min n1..nn}\index{min@{\tt min}}
&
Return the minimum of ${\tt n1} \cdots {\tt nn}$.
The list must contain at least 1 element.
\end{desctab}
For example:
\begin{verbatim}
${+ 1 2 3 4}, ${max 1 2 3 4}
\end{verbatim}
will produce:
\begin{verbatim}
10, 4
\end{verbatim}
\subsection{Boolean functions}
\nopagebreak
\par
\begin{desctab}
{\tt and b1..bn}\index{and@{\tt and}}
&
Return {\it true} \/if all of {\tt b1} {\ldots} {\tt bn} are {\it true},
or else {\it false}.
The {\tt and} of an empty list is {\it true}.
\\
{\tt not b}\index{not@{\tt not}}
&
Return {\it true} \/if {\tt b} is {\it false}, or else {\it true}.
\\
{\tt or b1..bn}\index{or@{\tt or}}
&
Return {\it true} \/if one of {\tt b1}
{\ldots} {\tt bn} is {\it true},
or else {\it false}.
The {\tt or} of an empty list is {\it false}.
\end{desctab}
For example:
\begin{verbatim}
.foreach a 0 1
.foreach b 0 1
.foreach c 0 1
$a $b $c | ${and $a $b $c} | ${or $a $b $c} | ${not $c}
.endforeach
.endforeach
.endforeach
\end{verbatim}
will produce:
\begin{verbatim}
0 0 0 | 0 | 0 | 1
0 0 1 | 0 | 1 | 0
0 1 0 | 0 | 1 | 1
0 1 1 | 0 | 1 | 0
1 0 0 | 0 | 1 | 1
1 0 1 | 0 | 1 | 0
1 1 0 | 0 | 1 | 1
1 1 1 | 1 | 1 | 0
\end{verbatim}
\subsection{String functions}
\par
\nopagebreak
\begin{desctab}
{\tt capitalize e..e}\index{capitalize@{\tt capitalize}}
&
Return the given words with the first character of each word converted
to upper case.
\\
{\tt strpad w len pw}\index{strpad@{\tt strpad}}
&
Return word {\tt w} made to length {\tt len} either by truncation or padding
at the end with copies of {\tt pw}.
\\
{\tt strlen w}\index{strlen@{\tt strlen}}
&
Return the number of characters in word {\tt w}.
\\
{\tt strindex c w}\index{strindex@{\tt strindex}}
&
Return the index of the first occurrence of character {\tt c} in word {\tt w},
counting from 1, or 0 if {\tt c} does not occur in {\tt w}.
\\
{\tt tolower e..e}\index{tolower@{\tt tolower}}
&
Return the given words with all upper case characters converted to lower case.
\\
{\tt toupper e..e}\index{toupper@{\tt toupper}}
&
Return the given words with all lower case characters converted to upper case.
\end{desctab}
For example:
\begin{verbatim}
.set s a list of WORDS
.set w bicycle
${len $s}, ${tolower $s}, ${toupper $s}, ${capitalize $s}
${strindex c $w}, ${strpad $w 15 -=}, ${strpad $w 3 -}
\end{verbatim}
will produce:
\begin{verbatim}
4, a list of words, A LIST OF WORDS, A List Of WORDS
3, bicycle-=-=-=-=, bic
\end{verbatim}
\subsection{List manipulation functions}
\par
\begin{desctab}
{\tt index w e..e}\index{index@{\tt index}}
&
Search for the first occurrence of word {\tt w} in the list of elements.
Return the index of that position counting from 1 upwards,
or 0 if {\tt w} does not occur in the list.
\\
{\tt member w e..e}\index{member@{\tt member}}
&
Return {\it true} \/if word \verb'w' occurs in the list of elements,
or {\it false} \/if it does not.
\\
{\tt shift e..e}\index{shift@{\tt shift}}
&
Return the given list with the first element deleted.
The {\tt shift} of an empty list is empty.
\\
{\tt first e..e}\index{first@{\tt first}}
&
Return the first element of the given list.
The first element of an empty list is empty.
\\
{\tt seplist s e..e}\index{seplist@{\tt seplist}}
&
Given a separator {\tt s} and a list of elements,
create a new string that
consists of all the elements in the list separated by a copy of {\tt s}.
{\em Note}:
the elements in the new list are {\em only} \/separated by the given separator,
no additional spaces are added, in contrast to other functions that
return a list.
\\
{\tt prefix pf e..e}\index{prefix@{\tt prefix}}
&
Given a prefix {\tt pf} and a list of elements,
create a new list that
consists of all the elements in the list prefixed by a copy of {\tt pf}.
\\
{\tt suffix sf e..e}\index{suffix@{\tt suffix}}
&
Given a suffix {\tt sf} and a list of elements,
create a new list that
consists of all the elements in the list suffixed by a copy of {\tt sf}.
\\
{\tt len e..e}\index{len@{\tt len}}
&
Return the number of elements in the given list.
\\
{\tt sort e..e}\index{sort@{\tt sort}}
&
Return a copy of the given list that is lexicographically sorted.
\\
{\tt rev e..e}\index{rev@{\tt rev}}
&
Reverse the elements in the list.
\end{desctab}
For example:
\begin{verbatim}
.set l a long list of nice and different words
index: ${index of $l}, ${index bla $l}
member: ${member of $l}, ${member bla $l}
shift: ${shift $l}
first: ${first $l}
seplist: ${seplist - $l}
prefix: ${prefix z $l}
suffix: ${suffix z $l}
len: ${len $l}
sort: ${sort $l}
rev: ${rev $l}
\end{verbatim}
will produce:
\begin{verbatim}
index: 4, 0
member: 1, 0
shift: long list of nice and different words
first: a
seplist: a-long-list-of-nice-and-different-words
prefix: za zlong zlist zof znice zand zdifferent zwords
suffix: az longz listz ofz nicez andz differentz wordsz
len: 8
sort: a and different list long nice of words
rev: words different and nice of list long a
\end{verbatim}
\subsection{Set operations}
\nopagebreak
These functions implement common operations on sets,
although they do not require sets as parameters.
However, if the input lists are not sets, some of the operations do not
result in a set.
\par
Note that {\tt comm} and {\tt excl} require a separator between
two groups of parameters, that cannot be used as a list element.
For this the empty string (`{\tt ""}') is chosen,
since it is unlikely that this will occur in a set.
However, this means that the empty string cannot occur in a set.
\par
\begin{desctab}
{\tt comm a "" b}\index{comm@{\tt comm}}
&
Both {\tt a} and {\tt b} are lists of elements.
Return a copy of all elements in {\tt a} that also occur in {\tt b}.
This can be used as a `set intersection' operation.
\\
{\tt excl a "" b}\index{excl@{\tt excl}}
&
Both {\tt a} and {\tt b} are lists of elements.
Return a copy of all elements in {\tt a} that do not occur in {\tt b}.
This can be used as a `set difference' operation.
\\
{\tt uniq e1..en}\index{uniq@{\tt uniq}}
&
Return a copy of the given list that is lexicographically sorted,
and where all duplicate elements are deleted.
This function can be used as a `set union' operation.
\\
\end{desctab}
For example:
\begin{verbatim}
.set a a b c d
.set b c d e f
{${comm $a "" $b}}, {${excl $a "" $b}}, {${uniq $a $b}}
\end{verbatim}
will produce:
\begin{verbatim}
{c d}, {a b}, {a b c d e f}
\end{verbatim}
\subsection{Regular expressions}
\nopagebreak
\par
The regular expressions in {\Tm} are based on the regular expressions
as used in {\Unix} shells.
A pattern must match the entire string.
The following meta-characters are recognized:
\par
\begin{desctab}
\verb'?' & Matches any character.  \\
\verb'\'
&
Matches the character following it.
It is used as an escape character for all
meta-characters, including itself. When used
in a set (see below), it is treated as an ordinary character.
\\
{\tt [set]}
&
Matches one of the characters in the set.
If the first character in the set is `\verb!^!',
it matches a character {\em not} \/in the set. A shorthand like
\verb!a-z! is used to specify a set of
characters \verb!a! up to \verb!z!, inclusive. The special
characters `\verb!]!' and `\verb!-!' have no special
meaning if they appear as the first characters in the set.
\par
Examples:
\par
\begin{tabular}{ll}
\verb![a-z]! & Any lower case alpha. \\
\verb![^]-]! & Any char except `\verb!]!' and `\verb!-!'. \\
\verb![^A-Z]! & Any char except upper case alpha. \\
\verb![a-zA-Z]! & Any alpha.
\end{tabular}
\\
\verb!*!
&
Matches the longest possible span of zero or more arbitrary characters.
\\
\verb!(form)!
&
Matches what {\tt form} matches,
and assigns the matching pattern to one of the numbered patterns.
The patterns are numbered from left to right by their opening bracket.
\end{desctab}
\par
When patterns are substituted the substitution string can also have
some meta-characters:

\begin{desctab}
\par
\verb!&! & Is replaced by the entire matched pattern.  \\
\verb!\d!
&
Is replaced by numbered pattern number `{\tt d}' as matched by
`{\tt (form)}'.
If the numbered pattern was not assigned in the original string,
it is empty. \verb!\0! is replaced by the entire matched pattern.
\\
\verb!\&! & Is replaced by `\verb!&!'.  \\
\verb!\\! & Is replaced by `\verb!\!'. \\
\end{desctab}
\par
\begin{desctab}
\\
{\tt filt ps pr e..e}\index{filt@{\tt filt}}
&
Given a search pattern {\tt ps} and a replacement pattern {\tt pr},
try to match all elements {\tt e} with {\tt ps},
and for each matching element return a copy of {\tt pr}
with the proper substitutions for any {\tt \&} and \verb!\d!.
\\
{\tt rmlist pat e..e}\index{rmlist@{\tt rmlist}}
&
Remove all elements matching {\tt pat} from the list.
\\
{\tt subs ps pr e..e}\index{subs@{\tt subs}}
&
Given a search pattern {\tt ps} and a replacement pattern {\tt pr},
copy all elements {\tt e},
and replace all elements matching {\tt ps} with {\tt pr}
with the proper substitutions for any {\tt \&} and \verb!\d!.
All elements that do not match {\tt ps} are copied without change.
\end{desctab}
For example:
\begin{verbatim}
.set l bla blup burp zwoing
[${filt b(*)p &(\1) $l}], [${rmlist b*p $l}]
[${subs b(*)p &(\1) $l}], [${subs b(*)p \&(\1) $l}]
\end{verbatim}
will produce:
\begin{verbatim}
[blup(lu) burp(ur)], [bla zwoing]
[bla blup(lu) burp(ur) zwoing], [bla &(lu) &(ur) zwoing]
\end{verbatim}
\subsection{Environment access functions}
\nopagebreak
\par
\begin{desctab}
{\tt getenv nm}\index{getenv@{\tt getenv}}
&
Return the value of environment variable {\tt nm}.
If the variable does not exist an empty string is returned.
\\
{\tt isinenv nm}\index{isinenv@{\tt isinenv}}
&
Determine whether environment variable {\tt nm} exists.
Return {\it true}\/ if it exists, or {\it false} \/if it does not.
\\
{\tt dsfilename}\index{dsfilename@{\tt dsfilename}}
&
Return the name of the file that describes the data structures.
\\
{\tt searchfile fn..fn}\index{searchfile@{\tt searchfile}}
&
Given a list of file names,
return a list
of full names files that have been located using the search path,
see the function {\verb+searchpath+}.
\par
If file can not be found in the search path,
the string \verb+?+ (a single question mark) is returned for that
file name.
\\
{\tt searchpath}\index{searchpath@{\tt searchpath}}
&
Return the search path for \verb+.include+ and \verb+.insert+ files.
By default,
the search path contains 
the directory the variable \verb+libpath+ is set to.
If additional
directories are specified with the \verb+-I+ option,
they are appended to the list.
\\
{\tt tplfilename}\index{tplfilename@{\tt tplfilename}}
&
Return the name of the current template file.
\\
{\tt tpllineno}\index{tpllineno@{\tt tpllineno}}
&
Return the current line number in the current template file.
\\
{\tt defined v}\index{defined@{\tt defined}}
&
Return {\it true} \/if variable {\tt v} is defined,
or {\it false} \/otherwise.
\end{desctab}
\par
For example:
\begin{verbatim}
${getenv SHELL}
\end{verbatim}
may produce (depending on the value of the \verb+SHELL+ environment variable):
\begin{verbatim}
/bin/csh
\end{verbatim}
\subsection{Data structure access functions}
\par
These functions are used to retrieve information about the data structure
file that was passed to {\Tm}.
\par
\begin{desctab}
{\tt typelist}\index{typelist@{\tt typelist}}
&
Return the list of types defined in the data structure file.
\\
{\tt inherits t}\index{inherits@{\tt inherits}}
&
Given a type {\tt t},
return the list of types from which this type inherits.
\\
{\tt ttypelist}\index{ttypelist@{\tt ttypelist}}
&
Return the list of {\em tuple} types defined in the data structure file.
\\
{\tt ctypelist}\index{ctypelist@{\tt ctypelist}}
&
Return the list of {\em constructor} \/types defined in the data structure file.
\\
{\tt telmlist t}\index{telmlist@{\tt telmlist}}
&
Given a tuple type {\tt t},
return the list of element names of that tuple.
If {\tt t} is a constructor type, an error message is given.
\\
{\tt ttypeclass t e}\index{ttypeclass@{\tt ttypeclass}}
&
Given a tuple type {\tt t}
and a tuple element name {\tt e},
return the type class of that element.
Possible type classes are {\tt single} and {\tt list}
for a single element and a list of elements respectively.
If {\tt t} is a constructor type, an error message is given.
\\
{\tt ttypellev t e}\index{ttypellev@{\tt ttypellev}}
&
Given a tuple type {\tt t} and a tuple element name {\tt e},
return return the level of list bracketing of the element. Thus,
an element of type \verb't' has list level 0, an element of type
\verb'[t]' has list level 1, and so on.
If {\tt t} is a constructor type, an error message is given.
\\
{\tt ttypename t e}\index{ttypename@{\tt ttypename}}
&
Given a tuple type {\tt t} and a tuple element name {\tt e},
return the type of that element.
If {\tt t} is a constructor type, an error message is given.
\\
{\tt conslist t}\index{conslist@{\tt conslist}}
&
Given a constructor type {\tt t},
return the list of constructors of that type.
If {\tt t} is a tuple type, an error message is given.
\\
{\tt celmlist t c}\index{celmlist@{\tt celmlist}}
&
Given a constructor type {\tt t} and a constructor name {\tt c},
return the list of element names of that constructor.
If {\tt t} is a tuple type, an error message is given.
\\
{\tt ctypeclass t c e}\index{ctypeclass@{\tt ctypeclass}}
&
Given a constructor type {\tt t},
a constructor {\tt c}
and a constructor element name {\tt e},
return the type class of that element.
Possible type classes are {\tt single} and {\tt list}
for a single element and a list of elements respectively.
If {\tt t} is a tuple type, an error message is given.
\\
{\tt ctypellev t c e}\index{ctypellev@{\tt ctypellev}}
&
Given a type {\tt t},
a constructor {\tt c}
and a constructor element name {\tt e},
return return the level of list bracketing of the element. Thus,
an element of type \verb't' has list level 0, an element of type
\verb'[t]' has list level 1, and so on.
If {\tt t} is a tuple type, an error message is given.
\\
{\tt ctypename t c e}\index{ctypename@{\tt ctypename}}
&
Given a type {\tt t},
a constructor {\tt c}
and a constructor element name {\tt e},
return the type of that element.
If {\tt t} is a tuple type, an error message is given.
\\
\end{desctab}
For example:
\begin{verbatim}
Constructors:
.foreach t ${ctypelist}
'$t' is a constructor type with constructors:
.foreach c ${conslist $t}
 '$c', with elements:
.foreach e ${celmlist $t $c}
  '$e', type '${ctypename $t $c $e}', list level ${ctypellev $t $c $e}
.endforeach
.endforeach
.endforeach

Tuples:
.foreach t ${ttypelist}
'$t' is a tuple type with elements:
.foreach e ${telmlist $t}
 '$e', type '${ttypename $t $e}', list level ${ttypellev $t $e}
.endforeach
.endforeach
\end{verbatim}
will produce (assuming the data structure definitions of page \pageref{plotds}):
\begin{verbatim}
Constructors:
'plot' is a constructor type with constructors:
 'XYPlot', with elements:
  'xycolor', type 'int', list level 0
  'points', type 'xypoint', list level 1
 'YPlot', with elements:
  'ycolor', type 'int', list level 0
  'xstart', type 'num', list level 0
  'xend', type 'num', list level 0
  'points', type 'num', list level 1

Tuples:
'xypoint' is a tuple type with elements:
 'x', type 'num', list level 0
 'y', type 'num', list level 0
\end{verbatim}
\subsection{Code generation service functions}
These functions implement some of the complicated operations that
are necessary during code generation. 
\par
One definition must be introduced:
A type $t$ {\em depends} \/on a type $s$ if $s$
is used in at least one of the elements of $t$,
or if $t$ has a type in one of its elements that depends on $s$.
Also,
a list of a type depends on that type.
Primitive types depend on nothing.
\par
For example:
in the data structure on page \pageref{plotds} type {\tt plot}
depends on the single types {\tt xypoint}, {\tt int}, and {\tt num},
and on the list types {\tt [xypoint]} and {\tt [num]}.
\par
Several functions in this group refer to the {\Tm} variables \verb+listpre+
and \verb+listsuff+ for a prefix and a suffix respectively to construct
a list name from a single type name.
If one of these variables is not defined, an empty string is assumed.
\par
\begin{desctab}
{\tt deptype g t1..tn}\index{deptype@{\tt deptype}}
&
Given a list of types {\tt t1} $\cdots$ {\tt tn},
return the list of {\em defined} types these types depend on.
If {\tt g} is {\tt single}, return the single types,
if {\tt g} is {\tt list}, return the list types.
For nested lists (lists with a list level higher than one),
the list name is mentioned plus all lower list levels.
The list names are constructed using the {\Tm} variables \verb+listpre+
and \verb+listsuff+.
\par
Thus, if one of the dependent types is a type \verb+foo+ with list
level 2, and the list types are requested, \verb+[foo]+ and
\verb+foo+ are returned (with suitable definitions of \verb+listpre+
and \verb+listsuff+ of course).
\\
{\tt depsort t1..tn}\index{depsort@{\tt depsort}}
&
Given a list of types {\tt t1} $\cdots$ {\tt tn},
rearrange them so that all tuple types that use a certain type
are placed after the type itself.
Thus, if a tuple type {\tt ta} uses a type {\tt tb}, {\tt tb} will
be placed before {\tt ta};
this is in effect a topological sort.
The function will complain about circularities.
\par
Using this function it is not necessary to introduce forward declarations
for tuple types and the manipulation functions on tuple types.
The function does not consider the types that are used in constructors,
since these may contain inherent circularities.
\\
{\tt mklist n e..e}\index{mklist@{\tt mklist}}
&
Given a list level {\tt n} and a list of type names {\tt e},
construct a list name for the given list level by prefixing {\tt n}
copies of the list prefix in variable \verb+listpre+ and suffixing
{\tt n} copies of the list suffix in the variable \verb+listsuff+.
\\
{\tt stemname e..e}\index{stemname@{\tt stemname}}
&
Given a list of list type names {\tt e},
return a list of stem names.
All pairs of list prefixes and suffices are stripped from all
the type names.
The list prefix is taken from the variable \verb+listpre+,
the list suffix is taken from the variable \verb+listsuff+.
If both are empty the list is returned unchanged.
\end{desctab}
For example:
\begin{verbatim}
.set listpre <
.set listsuff >
[${deptype single plot}], [${deptype single xypoint}]
[${mklist 3 a z}]
[${stemname <<a>> <<b> <c>> d <e>}]
\end{verbatim}
will produce (assuming the data structure definitions of page \pageref{plotds}):
\begin{verbatim}
[plot xypoint], [xypoint]
[<<<a>>> <<<z>>>]
[a <b c> d e]
\end{verbatim}
\subsection{Deferred evaluation}
\begin{desctab}
{\tt call m p1..pn}\index{call@{\tt call}}
&
Given a macro name \verb+m+ and a list of parameters \verb+p1..pn+,
invoke macro with the given list of parameters.
Before execution,
the formal parameters of the macro are set to the values in the parameter list.
To pass a list of words to a formal parameter,
it must be surrounded by double quotes.
The number of parameters must match the number of formal parameters,
or else an error message is given.
Note that each parameter word corresponds to one formal parameter.
Macros `see' the variable values and macros that are in effect at the moment
of execution of the macro.
All changes that are made to variables or macros within a macro
are invisible outside the macro.
During execution of the macro at least one \verb+.return+ command must be
given to record a return value for the macro.
The macro is not allowed to generate output, and therefore may only contain
line commands.
If you want to generate output, 
use the \verb+.call <macro> <parm>..<parm>+ form instead.
\\
{\tt eval e1..en}\index{eval@{\tt eval}}
&
Given a list of expressions \verb+e1..en+, evaluate all expressions,
and return them in a list.
\\
\end{desctab}
For example:
\begin{verbatim}
.set x $$[12+45]
x=$x, e = ${eval $x}
\end{verbatim}
will result in
\begin{verbatim}
x=$[12+45], e = 57
\end{verbatim}
\section{Arithmetic expressions}
\label{s.intexpr}
\index{arithmetic expressions|(}
Integer expressions have the form
\begin{verbatim}
$[<expr>]
\end{verbatim}
Each time the characters `{\tt \$[}' are encountered,
the string up to the next unbalanced `{\tt ]}' is evaluated,
and the resulting string is
interpreted as an integer expression.
The following operators are available, where the number in the first column
indicates priority; the lower the number, the stronger an operator binds:
\begin{quote}
\begin{tabular}{lll}
priority & symbol & function \\
0 & \verb'()' & Priority brackets. \\
1 & \verb'-' & Unary minus. \\
1 & \verb'!' & Boolean not. \\
1 & \verb'+' & Unary plus. \\
2 & \verb'*' & Multiplication. \\
2 & \verb'/' & Division. \\
2 & \verb'%' & Modulus. \\
3 & \verb'+' & Addition. \\
3 & \verb'-' & Subtraction.\\
4 & \verb'!=' & Not equal to. \\
4 & \verb'==' & Equal to. \\
4 & \verb'<=' & Less or equal. \\
4 & \verb'<' & Less. \\
4 & \verb'>=' & Greater or equal. \\
4 & \verb'>' & Greater. \\
5 & \verb'&' & Boolean and. \\
5 & \verb'|' & Boolean or. \\
\end{tabular}
\end{quote}
The integer `0' represents `false',
all other integers represent `true'.
Boolean functions always return `1' as `true'.
The operators $+$, $-$, $\%$, $*$ and $/$ are right binding.
That is,
\[ a\ {\rm op}\ b\ {\rm op}\ c = a\ {\rm op}\ (b\ {\rm op}\ c) \]
For example:
\begin{verbatim}
.set n 42
$[$n*$n], $[$n&1], $[1>2], $[00==0], $[2/2/2]
\end{verbatim}
will produce:
\begin{verbatim}
1764, 1, 0, 1, 2
\end{verbatim}
\index{arithmetic expressions|)}
\section{An example of a template}
Assuming that the file {\tt plot.ds} contains the data structure description
of page \pageref{plotds},
and the file {\tt macdemo.t} has the following contents:
\begin{flushleft}
\verbatiminput{macdemo.t}
\end{flushleft}
The command
\begin{verbatim}
    tm plot.ds macdemo.t
\end{verbatim}
will give the output:
\par
\begin{flushleft}
\verbatiminput{macdemo.out}
\end{flushleft}
