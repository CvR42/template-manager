\chapter{Invoking Tm}
\label{s.command}
\index{invoking Tm}
\index{command line}
\index{option}
In its simplest form, {\Tm} is invoked as follows:
\begin{verbatim}
tm <ds-file> <template-file>
\end{verbatim}
Where \verb'<ds-file>' is a data structure definition file, and
\verb'<template-file>' is a template file.
If no \verb'<template-file>' is given, input is read from standard input.
If neither file is given, an empty \verb'<ds-file>' is assumed, and
input is read from standard input.
\par
Also, the following flags can be given:

\begin{flushleft}
\begin{tabular}{ll}
\verb'-d<debugflags>'  & Set given debug flags. \\
\verb'-e<file>'        & Redirect errors to file \verb'<file>'. \\
\verb'-h'              & Show a help text. \\
\verb'-I<path>'        & Append \verb'<path>' to the search path. \\
\verb'-o<file>'        & Redirect output to file \verb'<file>'. \\
\verb'-s<var>'         & Set variable \verb'<var>' to the empty string. \\
\verb'-s<var>=<val>'   & Set variable \verb'<var>' to string \verb'<val>'. \\
\verb'-v'              & equivalent to \verb'-sverbose'. \\
\end{tabular}
\end{flushleft}

For the debugging flags, see the online help text given for
\begin{verbatim}
tm -h
\end{verbatim}
