\documentclass{article}
\usepackage{makeidx}
\usepackage{a4wide}
\usepackage{verbatim}
\def\Tm{{\sf Tm}}
\def\coil{{\bf Coil}}
\def\C{{\sf C}}
\def\Unix{{\bf Unix}}
\def\Miranda{Miranda}
\makeindex
\newenvironment{listing}{\small\begin{flushleft}}{\end{flushleft}}
\newenvironment{example}{\par}{}
\newcommand{\dfn}[1]{{\em {#1}}\index{#1}}
% Description environment: Don't add any space after previous item,
% indent left margin.
\newenvironment{desc}{\nopagebreak\vspace{-\bigskipamount}\vspace{-\parskip}\begin{list}{}{\setlength{\topsep}{0pt}\setlength{\rightmargin}{0pt}}\item[]}{\end{list}}
\newlength{\descwidth}
\setlength{\descwidth}{\textwidth}
\addtolength{\descwidth}{-3.4cm}
\newenvironment{desctab}{\begin{tabular*}{\textwidth}{l@{\extracolsep{\fill}}p{\descwidth}}}{\end{tabular*}}
\raggedbottom
\begin{document}
\title{The Tm Miranda template}
\author{Kees van Reeuwijk}
\maketitle
\section{Introduction}
\begin{figure}[h]
\verbatiminput{plot.ds}
\caption{\label{f.ds}An example of a data structure description file.}
\end{figure}
This section provides the necessary information to use {\Tm} for {\Miranda}.
In section \ref{sec:miraquickintro} a 'cookbook' introduction to the use
of the library is given, in section \ref{sec:miralib} the support
library is described in detail.
\par
The {\Miranda} support is provided by one file: {\tt mira.mt}.
The support library can generate the following code:
\begin{itemize}
\item
Data structure definitions.
\item
Input (scan) functions.
\item
Output (show) functions.
\end{itemize}
\par
Support is also provided for some primitive types.
\section{Usage}
\label{sec:miraquickintro}
You can use the following steps to obtain code from {\Tm}.
The steps are illustrated by an example.
\begin{enumerate}
\item
Design the data structure for your problem,
and encode it in a data structure description file.
For the example it is assumed that the file {\tt plot.ds} contains
the data structure definitions of Fig.~\ref{f.ds}.
\item
Create a configuration file that sets the following {\Tm} variables:
\begin{itemize}
\item
The variable {\tt wantdefs}: the names of the functions that you
will use in your program.
If these functions require other functions,
this will be deduced, and the necessary functions will be generated but
not mentioned in {\tt \%export} in the generated {\Miranda} source file.
\item
\begin{sloppypar}
The variable {\tt notwantdefs}:
in the rare case that {\Tm} deduces that a definition must be generated
while this is not desirable,
the name of this definition can be appended to the list of
{\Tm} variable {\tt notwantdefs}.
\end{sloppypar}
\item
The variable
{\tt also\_export}:
if next to the defined types and the requested functions other definitions
must be exported, list them in this variable.
See section \ref{sec:export} for details.
\end{itemize}
For the running example a file {\tt miraspec.t} is used with the
following contents:
\begin{listing}
\verbatiminput{miraspec.t}
\end{listing}
\item
Create a {\Tm} header file for the generated {\Miranda} source file.
This file should at least contain
\begin{itemize}
\item
An {\tt .insert} of the configuration file.
In this example {\tt miraspec.t}.
\item
An {\tt .include} of the library file {\tt mira.mt}.
\end{itemize}
The header file may also contain other stuff,
like definitions for primitive types.
For the example the header file is:
\begin{listing}
\verbatiminput{mirademo.t}
\end{listing}
\item
Add make rules to your makefile to generate the code:
\begin{listing}
\begin{verbatim}
mirademo.m : plot.ds mirademo.t miraspec.t
        tm plot.ds mirademo.t > mirademo.m
\end{verbatim}
\end{listing}
\end{enumerate}
\par
The file {\tt mirademo.m} that is generated from our
example configuration is listed
in appendix \ref{sec:democode}.
\section{The Miranda library}
\label{sec:miralib}
The {\Miranda} support library consists of one file: {\tt mira.mt}.
Note that you must set {\tt alldefs} or {\tt wantdefs} (and perhaps
{\tt notwantdefs}) to ensure that code is generated.
\subsection{Data structures}
{\Miranda} data structure definitions are generated from the {\Tm}
data structure definitions by leaving out the element names.
\subsection{Input}
Since the standard textual representation of {\Tm} data structures is
for a large number of data types equivalent with the representation used
in {\Miranda},
it is possible to use the {\Miranda} compiler itself to read the
standard data structures.
This can be done with the following trick:
Suppose there is a file {\tt data.tm} containing a data structure in the
internal representation,
and a file {\tt plot.m} containing {\Miranda} data type definitions
for the data to read,
the following {\Miranda} script will make the variable {\tt data}
equal to this data:
\begin{listing}
\verbatiminput{readin.m}
\end{listing}
\par
One problem with this trick is that any negative {\tt num} values
must be surrounded by brackets,
or else the {\Miranda} compiler will complain\footnote{
In all supported languages, the routines that are provided for writing
numbers will provide the brackets.
}.
\par
Alternatively a collection of scan functions can be generated to parse
a string into instances of the data structures.
These functions can be used to prevent repeated re-compilation of the
{\Miranda} scripts,
or if special data types are used in the data structure.
\begin{verbatim}
scan_<type> :: [char]->(<type>,[char])
\end{verbatim}
\begin{desc}
\index{scan_<type>@\verb+scan_<type>+}
Given a string to scan,
skip all leading spaces, tabs and comment from the given string,
scan one element of {\tt <type>} in the string,
and return a tuple consisting of that element and the remaining characters
in the scanned string.
If an error occurs during reading, {\tt error} is used.
\end{desc}
\subsection{Output}
{\Miranda} provides a {\tt show} function that should be capable of generating
the textual representation of {\Tm}.
However,
there are a few situation where it is impossible to use this
{\tt show} function:
\begin{itemize}
\item If special data types are used in the data structure.
\item If the generated output line is too long.
      Since {\tt show} will generate one very long output line,
      this might cause problems with some other software.
\end{itemize}
Therefore, alternative functions are provided.
\begin{verbatim}
show_<type> :: <type>->[char]
\end{verbatim}
\begin{desc}
\index{show_<type>@\verb+show_<type>+}
Given an element of {\tt <type>},
generate the textual representation for it.
\end{desc}
\subsection{{\Tm} and {\Miranda} configuration variables}
\label{sec:miraconfig}
The library uses a few {\Tm} variables to modify the contents
of the generated code.
Unless stated otherwise,
it is not necessary to set them;
a reasonable default will be chosen.
\par
\begin{desctab}
{\tt also\_export}\index{also_export@{\verb+also_export+}}
&
If set,
this variable lists additional functions that must be mentioned
in {\tt \%export}.
This is useful to {\tt \%export} functions that are defined in the input
file for {\Tm}.
It is also useful if local definitions of the generated file
must be made visible, see section \ref{sec:export}.
\\
{\tt wantdefs}\index{wantdefs@{\verb+wantdefs+}}
&
If set,
this variable contains the names of the type and function definitions that
must be generated.
If these functions require the definition of other functions,
the necessary functions will be generated,
but will not be mentioned in {\tt \%export}.
If {\tt alldefs} is set all possible definitions are generated.
\\
{\tt notwantdefs}\index{notwantdefs@{\verb+notwantdefs+}}
&
If set,
this variable contains the names of the type and function definitions for
which under all conditions {\tt no} code must be generated.
\\
{\tt alldefs}\index{alldefs@{\tt alldefs}}
&
If set,
this variable indicates that all possible type and function definitions must
be generated.
\\
\end{desctab}
\par
At least one of the variables {\tt wantdefs} and {\tt alldefs} must be set.
\subsection{Library functions for some special types}
For a few frequently used types the {\Miranda} support library can generate
the necessary code.
You can request generation by adding the required functions to the contents
of the {\Tm} variable {\tt wantdefs} or by setting {\tt alldefs}.
Note that if you set {\tt wantdefs}, {\Tm} will {\em not} deduce that
one of these functions is needed, you must request them.
The type specifications of the available functions are:
\begin{verbatim}
scan_bool :: [char]->(bool,[char])
scan_char :: [char]->(char,[char])
scan_num :: [char]->(num,[char])
scan_string :: [char]->([char],[char])
show_bool :: bool->[char]
show_char :: char->[char]
show_num :: num->[char]
show_string :: [char]->[char]
\end{verbatim}
\index{scan_bool@\verb+scan_bool+}
\index{scan_char@\verb+scan_char+}
\index{scan_num@\verb+scan_num+}
\index{scan_string@\verb+scan_string+}
\index{show_bool@\verb+show_bool+}
\index{show_char@\verb+show_char+}
\index{show_num@\verb+show_num+}
\index{show_string@\verb+show_string+}
It is assumed that
\begin{verbatim}
string == [char]
\end{verbatim}
But you must define this yourself (for example in the input file for {\Tm}).
\par
Note that the string representation used in {\Miranda} differs from the normal
representation: `\verb!\ddd!' with {\tt d} a digit is interpreted as
a {\em decimal} number in {\Miranda}, and as a {\em octal} number in
{\C} and several {\Unix} programs.
In the {\Tm} textual representation `\verb!\ddd!' is interpreted as
an {\em octal} number.
\subsection{Support for {\tt \%export}}
\label{sec:export}
\index{\%export@\verb+%%export+}
Without intervention the generated {\Miranda} file will contain a
{\tt \%export} directive for all requested functions and all required
data structures.
You can add your own definitions to this by listing them in {\Tm} variable
{\tt also\_export}.
This is useful to add functions that are defined in the input file
for {\Tm} to the {\tt \%export} list,
or to make local functions visible to other files.
\par
The following functions are guaranteed to be available,
exporting other local functions may be non-portable or may cause errors in
future versions of the {\Miranda} library.
\par
\begin{verbatim}
needcbrac :: num->[char]->[char]
\end{verbatim}
\begin{desc}
\index{needcbrac@\verb+needcbrac+}
Given a number {\tt n} and a string to scan,
ensure that there are {\tt n} close brackets possibly preceded by
spaces and comment.
Give an error if not all the brackets are found.
This function is intended to check that the brackets that have been
scanned by {\tt scanobrac} are balanced.
\end{desc}
\begin{verbatim}
scanconstr :: [char]->([char],[char])
\end{verbatim}
\begin{desc}
\index{scanconstr@\verb+scanconstr+}
Skip all leading spaces and comment,
and then match a string of letters or digits.
The matched string may be empty.
This is used to scan constructor names.
\end{desc}
\begin{verbatim}
scanlist :: ([char]->(*,[char]))->[char]->([*],[char])
\end{verbatim}
\begin{desc}
\index{scanlist@\verb+scanlist+}
Given a function to scan single element of a certain type and a string
to scan,
scan a list of elements of that type.
\end{desc}
\begin{verbatim}
scanobrac :: [char]->(num,[char])
\end{verbatim}
\begin{desc}
\index{scanobrac@\verb+scanobrac+}
Match any number of open brackets and preceding spaces and comments.
Return a tuple containing the number of matched brackets and the remaining
string.
To check that there is a matching number of close brackets the function
{\tt needcbrac} can be used.
\end{desc}
\begin{verbatim}
skipuntil :: char->[char]->[char]
\end{verbatim}
\begin{desc}
\index{skipuntil@\verb+skipuntil+}
Drop all elements from list {\tt x} up to and including element {\tt e}.
If {\tt x} does not contain {\tt e},
give an error message.
\end{desc}
\begin{verbatim}
stripspace :: [char]->[char]
\end{verbatim}
\begin{desc}
\index{stripspace@\verb+stripspace+}
Remove all leading comment, tabs, spaces and returns from the
given string.
\end{desc}
\begin{verbatim}
showlist :: (*->[char])->[*]->[char]
\end{verbatim}
\begin{desc}
\index{showlist@\verb+showlist+}
Given a function to show single element of a certain type and a list
of that type, show the list.
\end{desc}
\appendix
\section{Listing of the file `mirademo.m'}
\label{sec:democode}
\begin{listing}
\verbatiminput{mirademo.m}
\end{listing}
\printindex
\end{document}
