\chapter{Syntax of Tm data-structure descriptions}
\label{s.dsgram}
The data-structure description must have the following syntax:
\begin{grammar}
\nonterm{dslist} & \produces & \emptystring \bar \nonterm{ds} \nonterm{dslist}  \\
\\
\nonterm{ds} & \produces & \nonterm{name} \term{::=} \nonterm{inherits} \nonterm{constructorlist} \term{{\tt ;}} \\
 & \bar & \nonterm{name} \term{==} \nonterm{inherits} \term{(} \nonterm{tuple} \term{)} \term{;} \\
 & \bar & \term{\%include} \nonterm{string} \term{;} \\
\\
\nonterm{inherits} & \produces & \emptystring \bar \nonterm{name} \term{+} \nonterm{inherits} \\
\\
\nonterm{tuple} & \produces & \nonterm{element}  \\
& \bar & \nonterm{element} \term{,} \nonterm{tuple} \\
\\
\nonterm{constructorlist} & \produces & \nonterm{constructor}  \\
& \bar & \nonterm{constructor} \term{|} \nonterm{constructorlist}  \\
\\
\nonterm{constructor} & \produces & \nonterm{name} \nonterm{elementlist} \\
\\
\nonterm{elementlist} & \produces & \emptystring \\
& \bar & \nonterm{element} \nonterm{elementlist}  \\
\\
\nonterm{element} & \produces & \nonterm{name} \term{:} \nonterm{type} \\
\\
\nonterm{type} & \produces & \nonterm{name} \\
& \bar & \term{[} \nonterm{type} \term{]} \\
\end{grammar}
Characters in typewriter font ({\tt like this}) are tokens,
{\emptystring} stands for an empty token list,
\nonterm{name} stands for a token with the regular expression
\verb![a-zA-Z][a-zA-Z0-9]*!,
and \nonterm{string} stands for a string of arbitrary characters
surrounded by double quotes \verb+"+.
\par
If a type is defined repeatedly, the constructor or element lists
are concatenated. Redefinition of constructors is not allowed.
