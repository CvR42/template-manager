\chapter{An example project: expression optimization}
\label{s.calc}
In this chapter we will present a small but complete set of sources to parse
a simple language, and print it out in the standard Tm representation.
A second program is presented that reads in this representation,
replaces constant expressions with a single constant, and writes
it out again.

\section{The parser}
\label{s.calcparser}
You can use the following steps to obtain {\C} code from {\Tm}.
The steps are illustrated by an example:
\begin{enumerate}
\item
Design the data structures for your problem,
and encode them in a data structure description file.
For the example it is assumed that the file {\tt plot.ds} contains the
data structure definitions of page \pageref{plotds}.
\item
Create a configuration file that sets the following {\Tm} variables:
\begin{itemize}
\item
The variable {\tt basename}:
a name that can be used to identify the complete set of data structures
for which code will be generated. This name will be used for some global
routines such as \verb'stat_<basename>()' that are not specific to
a certain type or constructor.
\item
The variable {\tt wantdefs}: the names of the functions that you
will use in your program.
If these functions require other functions,
this will be deduced, and the necessary functions will be generated and
declared {\tt static} in the generated {\tt .c} file.
\item
The variable {\tt notwantdefs}:
in the rare case that {\Tm} deduces that a function must be generated
while this is not desirable,
the name of this function can be appended to the list of
variable {\tt notwantdefs}.
\end{itemize}
For the running example a file {\tt democonf.t} is used with the
following contents:
\begin{listing}
\verbatiminput{democonf.t}
\end{listing}
\item
Create a {\Tm} header file for the generated {\tt .c} and {\tt .h}
files.
These files should at least contain
\begin{itemize}
\item
A {\tt .insert} of the configuration file.
\item
A {\tt .include} of one of the template files.
\end{itemize}
The header file for the {\tt .c} file should contain a {\tt \#include}
for the generated {\tt .h} file,
and a {\tt \#include <tmc.h>}: an inclusion of the header file of the
{\C} support template.
The header files may also contain other stuff,
such as functions for primitive types.
For the example the two header files are:
\begin{listing}
\verbatiminput{demo.ht}
\end{listing}
and
\begin{listing}
\verbatiminput{demo.ct}
\end{listing}
\item
Add rules to your makefile to generate the code:
\begin{listing}
\begin{verbatim}
demo.c : plot.ds demo.ct democonf.t
        tm plot.ds demo.ct > demo.c

demo.h : plot.ds demo.ht democonf.t
        tm plot.ds demo.ht > demo.h
\end{verbatim}
\end{listing}
\end{enumerate}
\par
The files {\tt demo.h} and {\tt demo.c} that are generated from the
example configuration are listed
in appendices~\ref{s.demoh} and~\ref{s.democ}.
\section{The optimizer}
\section{Listing of the generated files}
