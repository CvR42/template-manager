\chapter{Introduction}
\label{s.intro}
\section{Motivation}
The transfer of structured data between programs is often carried out
using binary or ad-hoc textual formats.
However, this can result in ambiguous and non-portable file formats.
For example,
the {\Pascal} type declaration
\par
\begin{verbatim}
record foo x,y: integer; c: char; end;
\end{verbatim}
\par
is `equivalent' to the C type definition
\par
\begin{verbatim}
typedef struct { int x, y; char c; } foo;
\end{verbatim}
\par
This does not imply that it is easy to transfer data in these records
and structures from one language to the other.
Facilities that are provided for this purpose, like {\tt get} and
{\tt put} in Pascal and {\tt fread} and {\tt fwrite} in {\C} are useless,
and may even cause problems if data is transferred between different
implementations of the same language.
\par
An effective way to solve this problem is to introduce a textual
representation of the data.
The binary read and write routines can then be replaced by text printing and
parsing routines.
It is now necessary,
however, to define a suitable language for this representation;
if this is not done properly,
it may lead to inconsistency or system dependency.
\par
The program \defn{Tm} (for \defn{template manager}) allows such textual
representations to be defined in
a special data structure definition language.
{\Tm} is able to generate data structure definitions for a number of
programming languages from {\Tm} data structure definitions.
It can also generate code to read the textual representation into
internal data structures, and code to write these internal data structures
to the textual representation.
\par
At the moment, {\Tm} can generate code for {\C}, Lisp, {\Miranda} and Pascal.
Code generation for any other language is easily added,
because the generation is based on \defn{templates}:
source texts for the target programming language
interspersed with text substitution and repetition commands for {\Tm}.

\section{About this manual}
\begin{figure}
\begin{center}
\setlength{\unitlength}{1mm}
\begin{picture}(130,140)
%\put(0,0){\circle{1}}
%\put(130,0){\circle{1}}
%\put(130,140){\circle{1}}
%\put(0,140){\circle{1}}
\put(0,95){\framebox(60,45)[t]{\begin{minipage}{58\unitlength}\vspace{1mm}\begin{footnotesize}\verbatiminput{title.ds}\end{footnotesize}\end{minipage}}}
\put(70,95){\framebox(60,45)[t]{\begin{minipage}{58\unitlength}\vspace{1mm}\begin{footnotesize}\verbatiminput{title.ct}\end{footnotesize}\end{minipage}}}
\put(0,90){\makebox(60,5)[l]{\bf Data structure (.ds) file}}
\put(70,90){\makebox(60,5)[r]{\bf Template (.ct) file}}
%
\put(55,95){\vector(0,-1){10}}    % Ds to Tm
\put(75,95){\vector(0,-1){10}}    % template to Tm
%
\put(65,70){\oval(50,30)}
\put(40,55){\makebox(50,30){\Large \bf Tm}}
%
\put(65,55){\vector(0,-1){10}}    % Tm to .c
%
%
\put(35,5){\framebox(60,40)[t]{\begin{minipage}{58\unitlength}\vspace{1mm}\begin{footnotesize}\verbatiminput{title.c}\end{footnotesize}\end{minipage}}}
%%
%%
\put(35,0){\makebox(60,5){\bf Output (.c) file}}
\end{picture}

\end{center}
\caption{\label{f.flow}The flow of data through {\Tm}.
Given a data structure description file (a {\tt .ds} file) and
a template file (in this figure a {\tt .ct} file), {\Tm} produces
an output file (in this figure a {\tt .c} file).
}
\end{figure}
This document is a reference manual for template manager itself, and the
support for C. Templates for other languages are described in
separate documents. Since the templates for C are the most
extensive, and the most frequently used, they have been included
in this manual.
\par
This manual is not intended as a tutorial, but appendix \ref{s.calc}
shows the construction of two small programs that use the Tm templates.
Studying these will probably clarify a lot of the things that are
discussed there. It is therefore a very good idea to examine this example
before trying to understand the rest of the manual. The example programs
are available for downloading; compiling and perhaps modifying the code
is also very useful.
\par
Fig.~\ref{f.flow} shows the flow of data through {\Tm}. The syntax and
meaning of the data structure
definition file is described in section~\ref{s.ds}.
The syntax and meaning of the template file is described in
section~\ref{s.tm}.
Section~\ref{s.tmc} describes the C template and support library.
\section{Terminology}
A few notes on the used terminology:
{\Tm} is given a data structure file and a file
---called the {\em template file}\index{template file}---
containing source code and text substitution commands.
From these files a source file for the target language is generated.
This is called {\em translation}\index{translation}.
Target languages are supported by providing {\em standard templates} \/for
them.
\section{Used abbreviations}
In this manual the following shorthand notations are used:
\par
\index{abbreviations|(}
\begin{flushleft}
\begin{tabular}{ll}
{\tt <type>} \index{type@{\tt <type>}} & A type.  \\
{\tt <fieldtype>} \index{fieldtype@{\tt <fieldtype>}} & The type of a field.  \\
{\tt <fieldname>} \index{fieldname@{\tt <fieldname>}} & The name of a field.  \\
{\tt <something>} \index{something@{\tt <something>}} & The actual contents of this fragment are not important.  \\
{\tt <basename>} \index{basename@{\tt <basename>}} & Fill in the value of {\Tm} variable {\tt basename}. \\
\end{tabular}
\end{flushleft}

