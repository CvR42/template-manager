\chapter{Syntax of Tm data-structure descriptions}
\label{s.dsgram}
The data structure description must have the following syntax:
\begin{production}{dsList}
\emptystring \\
ds dsList \\
\end{production}

\begin{production}{ds}
name \term{::=} inherits constructorList \term{;} \\
name \term{==} inherits \term{(} tuple \term{)} \term{;} \\
name \term{=} classcomponentList \term{;} \\
name \term{~=} classcomponentList \term{;} \\
name \term{->} type \term{;} \\
\term{\%include} string \term{;} \\
\end{production}

\begin{production}{inherits}
\emptystring \\
name \term{+} inherits \\
\end{production}

\begin{production}{tuple}
field \\
field \term{,} tuple \\
\end{production}

\begin{production}{constructorList}
constructor  \\
constructor \term{|} constructorList  \\
\end{production}

\begin{production}{constructor}
name fieldList \\
\end{production}

\begin{production}{fieldList}
\emptystring \\
field fieldList  \\
\end{production}

\begin{production}{classcomponentList}
\emptystring \\
classcomponent \\
classcomponent \term{+} classcomponentList \\
\end{production}

\begin{production}{classcomponent}
\term{\{} tuple \term{\}} \\
name \\
subclassList \\
\term{(} classcomponentList \term{)} \\
\end{production}

\begin{production}{subclassList}
subclass \\
subclass \term{|} subclassList \\
\end{production}

\begin{production}{subclass}
name \term{:} classcomponent \\
\end{production}

\begin{production}{field}
name \term{:} type \\
\end{production}

\begin{production}{type}
name \\
\term{[} type \term{]} \\
\end{production}

Characters in typewriter font \term{like this} are tokens,
{\emptystring} stands for an empty token list,
\nonterm{name} stands for a token with the regular expression
\verb![a-zA-Z][a-zA-Z0-9]*!,
and \nonterm{string} stands for a string of arbitrary characters
surrounded by double quotes \verb+"+.
\par
If a constructor base type is defined repeatedly, the definitions
are merged. Redefinition of all other types is not allowed.
