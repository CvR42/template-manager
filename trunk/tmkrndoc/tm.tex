\chapter{The text substitution language}
\label{s.tm}
The standard modules for the various languages are implemented
with the text substitution language described in this section.
\par
One should only use the text substitution language directly if the available
modules do not meet the requirements,
since writing in this language is not exactly a pleasure.
The casual user of {\Tm} is expected to use only the standard modules that
are included in the {\Tm} package.
For this purpose it is sufficient to know
the {\tt .include} command (to invoke the modules),
the {\tt .insert} command (to insert configuration files),
the {\tt .set} command (to set some parameters for the modules),
and the {\tt ..} command (to add comments to the {\Tm} input).
\section{Conventions}
Boolean functions return the string `{\tt 0}' for
{\it false} or the string `{\tt 1}' for {\it true}.
At places where a boolean expression is expected,
the string `{\tt 0}' is {\it false},
all other strings are {\it true}.
\par
A \defn{number} is a string of digits with an optional sign before it.
\par
A \defn{word} is either a string of non-blank characters, or a string of
arbitrary characters surrounded by double quotes. A quoted string
must end before the end of the line, and it may not contain
double quote characters.
Thus,
the following lines each contain one word:
\begin{showfile}
\begin{verbatim}
a
word
"this is quoted and therefore treated as one word"
\end{verbatim}
\end{showfile}
\section{Line commands}
Each time a dot (`{\tt .}') is encountered as the first character of a line,
that line is interpreted as a {\Tm} line command.
If the second character of the line is also a dot,
the line is comment, and is completely ignored.
Otherwise the first word after the dot
(possibly preceded by some spaces and tabs)
is interpreted as the line command,
and the remainder of the line is interpreted as a list of parameters.
The list of parameters is evaluated each time the command is executed.
Evaluation of the parameters does {\em not} occur for termination commands
({\tt .endif}, {\tt .endforeach}, {\tt .endwhile}, etc.).
\subsection{Assignment}
\label{s.assignment}
Assignment statements modify the value of a variable. In most of these
statements, the new value is stored in the current \defn{context}.
A context is created upon invocation of a macro, or the execution of a
\verb'.include' command. Upon completion of the macro, or termination
of the \verb'.include', the context is destroyed, and the old value of
a variable (if it has a value in the old context) is visible again.
This means that it usually is not necessary to worry about side-effects
of macros or included files.
\par
Since in some cases it is desirable to allow side-effects, the
\verb'.globalappend' and \verb'.globalset' commands are provided.
These commands remove all current values of a variable from all
contexts, and store the new value in the outermost context. As a result
the new value is visible in all contexts.
\begin{verbatim}
.set <var> <val>..<val>
\end{verbatim}
\begin{desc}
\index{.set@\verb+.set+}
Set variable {\tt <var>} to the given list of values.
The new value of the variable is visible in the current context.
\end{desc}
\begin{verbatim}
.append <var> <val>..<val>
\end{verbatim}
\begin{desc}
\index{.append@\verb+.append+}
Append the given list of values after the current values of variable
{\tt <var>}.
The new value of the variable is visible in the current context.
For example:
\begin{showfile}
\verbatiminput{append.xt}
\end{showfile}
will produce:
\begin{showfile}
\verbatiminput{append.out}
\end{showfile}
\end{desc}
\begin{verbatim}
.globalappend <var> <val> <val> <val>
\end{verbatim}
\begin{desc}
Append the given list of values after the current value of variable
\verb'<var>'. The new value of the variable visible in {\em all}
variable contexts.
\end{desc}
\begin{verbatim}
.globalset <expr>
\end{verbatim}
\begin{desc}
Set variable \verb'<var>' to the given list of values.  The new value
of the variable visible to {\em all} variable contexts.
\end{desc}
\subsection{Datastructure modification}
These commands directly modify the data structures that have been
read in from the data structure file. There is no way to revert the
changes.
\begin{verbatim}
.rename <old> <new>
\end{verbatim}
\begin{desc}
\index{.rename@\verb+.rename+}
Rename all occurences of type \verb'<old>' to type \verb'<new>' at
all places where types occur: in type definitions, inheritances,
and fields.

It is not allowed to rename a defined type to another defined type,
nor is it allowed to introduce inheritance cycles.
\end{desc}
\begin{verbatim}
.deletetype <t>..<t>
\end{verbatim}
\begin{desc}
\index{.deletetype@\verb+.deletetype+}
Delete the definitions of all given types. Deletion requests for
non-existing types is silently ignored.
\end{desc}
For example, these commands can be used to remove alias types from
data structures. Given the following data structure definition file
\begin{showfile}
\verbatiminput{alias.ds}
\end{showfile}
The following template
\begin{showfile}
\verbatiminput{unalias.xt}
\end{showfile}
Will produce the following output.
\begin{showfile}
\verbatiminput{unalias.out}
\end{showfile}
\subsection{Flow control}
\begin{verbatim}
.exit <n>
\end{verbatim}
\begin{desc}
\index{.exit@\verb+.exit+}
Stop {\Tm} immediately with exit code {\tt <n>}.
This is usually used to halt upon fatal errors in a template.
\end{desc}
\begin{verbatim}
.foreach <var> <val>..<val>
\end{verbatim}
\begin{desc}
\index{.foreach@\verb+.foreach+}
Set variable {\tt <var>} to each of the values {\tt <val>} and do a
translation of all following lines up to the next unbalanced
{\tt .endforeach} for each of these values of {\tt <var>}.
The new value of the variable is only visible in the current context.
See section~\ref{s.assignment} for a discussion of contexts.
\par
For example:
\begin{showfile}
\verbatiminput{foreach.xt}
\end{showfile}
will produce:
\begin{showfile}
\verbatiminput{foreach.out}
\end{showfile}
\end{desc}
\begin{verbatim}
.if <expr>
\end{verbatim}
\begin{desc}
\index{.if@\verb+.if+}
Evaluate {\tt <expr>}.
If {\tt <expr>} is {\it true},
translate the following lines up to the next unbalanced
{\tt .else} or {\tt .endif}
and if relevant skip all lines between {\tt .else} and {\tt .endif}.

If {\tt <expr>} is {\it false},
only translate all lines between the following unbalanced
{\tt .else} and {\tt .endif}.
If no {\tt .else} is encountered, no translation is done.
\par
For example:
\begin{showfile}
\verbatiminput{if.xt}
\end{showfile}
will produce:
\begin{showfile}
\verbatiminput{if.out}
\end{showfile}
\end{desc}
\begin{verbatim}
.switch <expr>
.case <expr> .. <expr>
.default
.endswitch
\end{verbatim}
\begin{desc}
Evaluate the parameter of the \verb'.switch' expression; the result
should be a single word. Evaluate all \verb'.case' lines from top
to bottom, and if the list of words of a \verb'.case' line contains
the word that is switched on, execute the lines of that case.
Execution stops at the next \verb'.case' or \verb'.default' line,
or at the \verb'.endswitch' line.

If the list of cases is exhausted, but no case has been executed,
execute the statement in the \verb'.default' section, if it exists.

The list of words of a case statement may contain any number of words,
including no words at all. There can be at most one \verb'.default'
section. Note that more than one case can be executed for a switch value.
\par
For example:
\begin{showfile}
\verbatiminput{switch.xt}
\end{showfile}
will produce:
\begin{showfile}
\verbatiminput{switch.out}
\end{showfile}
\end{desc}
\begin{verbatim}
.while <expr>
\end{verbatim}
\begin{desc}
\index{.while@\verb+.while+}
Evaluate {\tt <expr>} and translate the subsequent lines up to the next
unbalanced {\tt .endwhile} if {\tt <expr>} is {\it true}.
Repeat this until {\tt <expr>} is {\it false}.
\end{desc}
\subsection{File inclusion and redirection}
\begin{verbatim}
.error <line>
\end{verbatim}
\begin{desc}
\index{.error@\verb+.error+}
Write {\tt <line>} to the error output stream.
This is useful for debugging or error messages.
\end{desc}
\begin{verbatim}
.include <filename>
\end{verbatim}
\begin{desc}
\index{.include@\verb+.include+}
Use the {\Tm} commands in {\tt <filename>} as if they were inserted at
this position.
It is not allowed to leave commands unterminated in {\tt <filename>}.
Nested {\tt .include} or {\tt .insert} commands can occur to an arbitrary depth.
No checking is done on recursion!
All changes to variables that are done in this module are invisible after
the {\tt .include} has terminated.
\par
Before a \verb'.include' file is executed, a new variable context is
created. This variable context is destroyed upon completion of execution
of the file. This behaviour is useful for the invocation of standard
translation modules.
\par
If the file cannot be opened, {\Tm} searches a list of directories for it.
By default this list contains the directory of the variable
\verb+libpath+, but the user may supply additional directories with the
\verb+-I+ command line option.
\end{desc}
\begin{verbatim}
.insert <filename>
\end{verbatim}
\begin{desc}
\index{.insert@\verb+.insert+}
Use the {\Tm} commands in {\tt <filename>} as if they were inserted at
this position.
It is not allowed to leave commands unterminated in {\tt <filename>}.
Nested {\tt .include} or {\tt .insert}
commands can occur to an arbitrary depth.
No checking is done on recursion!
\par
The variable settings that are done in an {\tt .insert}
file remain visible outside the file.
This is useful for configuration files or other files that
must make global changes to variables.
\par
If the file cannot be opened, {\Tm} searches a list of directories for it.
By default this list contains the directory of the variable
\verb+libpath+, but the user may supply additional directories with the
\verb+-I+ command line option.
\end{desc}
\begin{verbatim}
.redirect <filename>
\end{verbatim}
\begin{desc}
\index{.redirect@\verb+.redirect+}
Redirect the output up to the balancing \verb'.endredirect' to the given
file.
The parameter list of the \verb'.redirect' command must evaluate to
exactly one filename.
The old contents of the file is overwritten without warning.
\end{desc}
\begin{verbatim}
.appendfile <filename>
\end{verbatim}
\begin{desc}
\index{.appendfile@\verb+.appendfile+}
Append the output up to the balancing \verb'.endappendfile' to the given
file. The parameter list of the \verb'.appendfile' command must evaluate
to exactly one filename. If the file does not exist it is created.
\end{desc}
\subsection{Macros}
\begin{verbatim}
.call <macro> <parm>..<parm>
\end{verbatim}
\begin{desc}
Invoke macro \verb+<macro>+ with the given list of parameters. Before
execution of the macro, the formal parameters of the macro are set to
the values in the parameter list. The number of parameters must match the
number of formal parameters, or else an error message is given. To pass
a list of words to a formal parameter, surround it with double quotes.
Before a macro is executed, a new execution context is created. This
context is destroyed upon completion of the macro.
\par
During execution of the macro, \verb+.return+ commands may be given to
record a return value for the macro, but in the \verb+.call+ command this
value is ignored.  If you want to use it, use the
\verb+${call <macro> <parm>..<parm>}+ form instead.
\par
For example:
\begin{showfile}
\verbatiminput{linemacro.xt}
\end{showfile}
will produce:
\begin{showfile}
\verbatiminput{linemacro.out}
\end{showfile}
\end{desc}
\begin{verbatim}
.macro <name> <parm>..<parm>
\end{verbatim}
\begin{desc}
Define a new macro \verb+<name>+ with the given list of formal parameters.
Any previous macro of this name is replaced by the new definition.
The body of the macro consists of the lines after the
\verb+.macro+ up to the next unbalanced \verb+.endmacro+.
A macro may contain any {\Tm} command, including other \verb+.macro+
definitions.
Macros may return a list of words (see the \verb+.return+ command),
and may be invoked with the line command \verb+.call+ or an expression
of the form \verb+${call <macro> parm..parm}+.
See the description of these commands for further details.
\par
For example (a \verb'$[]' expression evaluates a numerical expression):
\begin{showfile}
\verbatiminput{macro.xt}
\end{showfile}
will produce:
\begin{showfile}
\verbatiminput{macro.out}
\end{showfile}
Note that the variable \verb'n' retains the value it had before the
invocations of the macro, because all changes to \verb'n' that are
done in the macro are removed upon completion of the macro.
Only by using the \verb'.globalset' or \verb'.globalappend' command
the change can be made permanent.
\end{desc}
\begin{verbatim}
.return <val>..<val>
\end{verbatim}
\begin{desc}
Record the given values as return value of the current macro.
Any previous return value is replaced by this one.
Contrary to the {\tt return} command in {\C}, this \verb+.return+
command does {\em not} terminate execution of the macro.
\par
See the example shown for the \verb'.macro' command for an example of the
use of the \verb'.return' command.
\end{desc}
\section{Variables}
Variables are referenced with the expression
{\tt \$(<varname>)} or optionally {\tt \$<letter>} for single character
variables.
Each time the characters `{\tt \$(}' are encountered,
the evaluated string up to the next unbalanced `{\tt )}' 
is interpreted as the name of a variable.
If `{\tt \$<letter>}' or `{\tt \$<digit>}' is encountered it is interpreted as
a reference to a variable with name `{\tt <letter>}' or {\tt <digit>}
respectively.
\par
For example:
\begin{showfile}
\verbatiminput{var.xt}
\end{showfile}
will produce:
\begin{showfile}
\verbatiminput{var.out}
\end{showfile}
Note that the result for `\verb+$bla+' is `{\tt +la}', since the
variable `{\tt b}' is referenced, not the variable `{\tt bla}'.
\par
If the characters `{\tt \$\{}' are encountered the evaluated string up to
the next unbalanced `{\tt \}}' is interpreted as a function invocation,
and the result of that function replaces the invocation string,
see section~\ref{s.fn}.
If the characters `{\tt \$[}' are encountered the evaluated string up
to the next unbalanced `{\tt ]}' is interpreted as an integer expression,
and the result of the evaluation of that expression replaces the expression,
see section~\ref{s.intexpr}.
If any other character is encountered after the `{\tt \$}',
it is copied literally.
\par
The following variables are predefined:
\par
\begin{desctab}
{\tt libpath}\index{libpath@{\tt libpath}}
&
The absolute directory path of the library of standard modules.
This may differ between installations,
so use this variable to locate the library, or let {\Tm} locate
it using its search path.
By default Tm uses a built-in path, but you can change this by setting
the environment variable TMLIBPATH\index{TMLIBPATH@\verb'TMLIBPATH'}.
\\
{\tt pathsep}\index{pathsep@{\tt pathsep}}
&
The character used as separation character between elements in a directory
path. This is operating system dependent.
\\
{\tt kernel-version}\index{kernel-version@{\tt kernel-version}}\index{version!Tm kernel version number}
\index{Tm kernel!version number}
&
The version number of the {\Tm} kernel, of which {\Tm} itself is a part.
\\
{\tt tmvers}\index{tmvers@{\tt tmvers}}\index{version!Tm version number}
\index{Tm!version number}
&
The version number of {\Tm} itself.
Not that this version number is different from the version number of the
kernel, of which {\Tm} is a part.
{\tt tmvers} is an integer, and is used in templates to determine whether
the version of {\Tm} that is executing the template is recent enough.
{\tt tmkernel-version} is a string, and is used to label the generated
code.
\\
{\tt verbose}\index{verbose@{\tt verbose}}
&
The verbose flag.
It is {\it true} if the flag {\tt -v} has been given at the command line,
else {\it false}.
\end{desctab}
\par
The variables {\tt listpre} and {\tt listsuff} are used by the functions
\verb+mklist+, \verb+stemname+ and \verb+deptype+ to construct list type
names from elements names or vice versa.
\section{Functions}
\label{s.fn}
Functions are invoked with the expression
\begin{showfile}
\begin{verbatim}
${<fnname> <par>..<par>}
\end{verbatim}
\end{showfile}
Each time the characters `{\tt \$\{}' are encountered,
the string up to the next unbalanced `{\tt \}}' is evaluated,
and the resulting string is
interpreted as a function invocation consisting of a function with the name
{\tt <fnname>} possibly followed by a number of parameter words.
\subsection{Comparison functions}
Most comparison functions only work on numbers---and will complain if they
get anything else than a number---but {\tt eq} and {\tt neq} work on strings.
Do {\it not} compare numbers with these functions
since this is not safe (to these functions `{\tt 0}'
is different from `{\tt 00}').
\par
\begin{desctab}
{\tt eq sa sb}\index{eq@{\tt eq}}
&
Return {\it true} if the string $\verb'sa' = \verb'sb'$.
Strings are equal if they are of the same length and are have the
same character at the same position in the string.
To compare numbers use {\tt ==}.
\\
{\tt neq sa sb}\index{neq@{\tt neq}}
&
Return {\it true} if the string ${\tt sa} \not= {\tt sb}$.
To compare numbers use \verb'!='.
\\
{\tt == na nb}\index{==@{\tt ==}}
&
Return {\it true} if ${\tt na} = {\tt nb}$.
To compare strings use {\tt eq}.
\\
\verb'!= na nb'\index{"!=@{\tt "!=}}
&
Return {\it true} if ${\tt na} \not= {\tt nb}$.
To compare strings use {\tt neq}.
\\
{\tt < na nb}\index{<@{\tt <}}
&
Return {\it true} if ${\tt na} < {\tt nb}$.
\\
{\tt <= na nb}\index{<=@{\tt <=}}
&
Return {\it true} if ${\tt na} \leq {\tt nb}$.
\\
{\tt > na nb}\index{>@{\tt >}}
&
Return {\it true} if ${\tt na} > {\tt nb}$.
\\
{\tt >= na nb}\index{>=@{\tt >=}}
&
Return {\it true} if ${\tt na} \geq {\tt nb}$.
\end{desctab}
For example:
\begin{showfile}
\verbatiminput{cmp.xt}
\end{showfile}
will produce:
\begin{showfile}
\verbatiminput{cmp.out}
\end{showfile}
\subsection{Arithmetic functions}
All arithmetic functions only work on numbers.
\par
\nopagebreak
\begin{desctab}
{\tt + n1..nn}\index{+@{\tt +}}
&
Return ${\tt n1} + \cdots + {\tt nn}$.
The sum of an empty list is 0.
\\
{\tt - na nb}\index{-@{\tt -}}
&
Return ${\tt na} - {\tt nb}$.
\\
{\tt * n1..nn}\index{*@{\tt *}}
&
Return ${\tt n1} \times \cdots \times {\tt nn}$.
The product of an empty list is 1.
\\
{\tt / na nb}\index{/@{\tt /}}
&
Return ${\tt na} / {\tt nb}$.
\\
{\tt \% na nb}\index{"%@\verb+%+}
&
Return the remainder of the division of {\tt na} by {\tt nb}.
\\
{\tt max n1..nn}\index{max@{\tt max}}
&
Return the maximum of ${\tt n1} \cdots {\tt nn}$.
The list must contain at least 1 element.
\\
{\tt min n1..nn}\index{min@{\tt min}}
&
Return the minimum of ${\tt n1} \cdots {\tt nn}$.
The list must contain at least 1 element.
\end{desctab}
For example:
\begin{showfile}
\verbatiminput{arith.xt}
\end{showfile}
will produce:
\begin{showfile}
\verbatiminput{arith.out}
\end{showfile}
\subsection{Boolean functions}
\nopagebreak
\par
\begin{desctab}
{\tt and b1..bn}\index{and@{\tt and}}
&
Return {\it true} if all of {\tt b1} {\ldots} {\tt bn} are {\it true},
or else {\it false}.
The {\tt and} of an empty list is {\it true}.
\\
{\tt not b}\index{not@{\tt not}}
&
Return {\it true} if {\tt b} is {\it false}, or else {\it true}.
\\
{\tt or b1..bn}\index{or@{\tt or}}
&
Return {\it true} if one of {\tt b1}
{\ldots} {\tt bn} is {\it true},
or else {\it false}.
The {\tt or} of an empty list is {\it false}.
\\
{\tt if cond a b}
&
Evaluate the expression {\tt cond}, and if it is {\it true}, return
{\tt a}, else return {\tt b}.
If {\tt a} or {\tt b} are absent, an empty string is assumed.
\end{desctab}
For example:
\begin{showfile}
\verbatiminput{boolean.xt}
\end{showfile}
will produce:
\begin{showfile}
\verbatiminput{boolean.out}
\end{showfile}
\subsection{String functions}
\nopagebreak
\begin{desctab}
{\tt capitalize e..e}\index{capitalize@{\tt capitalize}}
&
Return the given words with the first character of each word converted
to upper case.
\\
{\tt strpad w len pw}\index{strpad@{\tt strpad}}
&
Return word {\tt w} made to length {\tt len} either by truncation or padding
at the end with copies of {\tt pw}.
\\
{\tt strlen w}\index{strlen@{\tt strlen}}
&
Return the number of characters in word {\tt w}.
\\
{\tt strindex c w}\index{strindex@{\tt strindex}}
&
Return the index of the first occurrence of character {\tt c} in word {\tt w},
counting from 1, or 0 if {\tt c} does not occur in {\tt w}.
\\
{\tt tolower e..e}\index{tolower@{\tt tolower}}
&
Return the given words with all upper case characters converted to lower case.
\\
{\tt toupper e..e}\index{toupper@{\tt toupper}}
&
Return the given words \verb'e' with all lower case characters converted to upper case.
\\
{\tt tr o n e..e}\index{tr@{\tt tr}}
&
Given a string of old characters \verb'o', a list of new characters \verb'n',
and a list of words \verb'e',
return all \verb'e', with all characters occuring in \verb'o' replaced
with the corresponding character in \verb'n'.
The strings \verb'o' and \verb'n' must be of the same length.
\end{desctab}
For example:
\begin{showfile}
\verbatiminput{textfn.xt}
\end{showfile}
will produce:
\begin{showfile}
\verbatiminput{textfn.out}
\end{showfile}
\subsection{List manipulation functions}
\par
\begin{desctab}
{\tt index w e..e}\index{index@{\tt index}}
&
Search for the first occurrence of word {\tt w} in the list of elements.
Return the index of that position counting from 1 upwards,
or 0 if {\tt w} does not occur in the list.
\\
{\tt member w e..e}\index{member@{\tt member}}
&
Return {\it true} if word \verb'w' occurs in the list of elements,
or {\it false} if it does not.
\\
{\tt shift e..e}\index{shift@{\tt shift}}
&
Return the given list with the first element deleted.
The {\tt shift} of an empty list is empty.
\\
{\tt first e..e}\index{first@{\tt first}}
&
Return the first element of the given list.
The first element of an empty list is empty.
\\
{\tt seplist s e..e}\index{seplist@{\tt seplist}}
&
Given a separator {\tt s} and a list of elements, create a new string
that consists of all the elements in the list separated by a copy of
{\tt s}.  {\em Note}: in contrast to other functions that return a list,
the elements in the new list are {\em only} separated by the given
separator; no additional spaces are added.
\\
{\tt prefix pf e..e}\index{prefix@{\tt prefix}}
&
Given a prefix {\tt pf} and a list of elements,
create a new list that
consists of all the elements in the list prefixed by a copy of {\tt pf}.
\\
{\tt suffix sf e..e}\index{suffix@{\tt suffix}}
&
Given a suffix {\tt sf} and a list of elements,
create a new list that
consists of all the elements in the list suffixed by a copy of {\tt sf}.
\\
{\tt len e..e}\index{len@{\tt len}}
&
Return the number of elements in the given list.
\\
{\tt sort e..e}\index{sort@{\tt sort}}
&
Return a copy of the given list that is lexicographically sorted.
\\
{\tt sizesort e..e}\index{sizesort@{\tt sizesort}}
&
Return a copy of the given list that is sorted from smallest to largest
word. If two words are equally long, they are sorted lexicographically.
\\
{\tt rev e..e}\index{rev@{\tt rev}}
&
Reverse the elements in the list.
\end{desctab}
For example:
\begin{showfile}
\verbatiminput{listfn.xt}
\end{showfile}
will produce:
\begin{showfile}
\verbatiminput{listfn.out}
\end{showfile}
\subsection{Set operations}
\nopagebreak
These functions implement common operations on sets,
although they do not require sets as parameters.
However, if the input lists are not sets, some of the operations do not
result in a set.
\par
Note that {\tt comm} and {\tt excl} require a separator between
two groups of parameters, that cannot be used as a list element.
For this the empty string (`{\tt ""}') is chosen,
since it is unlikely that this will occur in a set.
However, this means that the empty string cannot occur in a set.
\par
\begin{desctab}
{\tt comm a "" b}\index{comm@{\tt comm}}
&
Both {\tt a} and {\tt b} are lists of elements.
Return a copy of all elements in {\tt a} that also occur in {\tt b}.
This can be used as a `set intersection' operation.
\\
{\tt excl a "" b}\index{excl@{\tt excl}}
&
Both {\tt a} and {\tt b} are lists of elements.
Return a copy of all elements in {\tt a} that do not occur in {\tt b}.
This can be used as a `set difference' operation.
\\
{\tt uniq e1..en}\index{uniq@{\tt uniq}}
&
Return a copy of the given list that is lexicographically sorted,
and where all duplicate elements are deleted.
This function can be used as a `set union' operation, and to convert
an arbitrary list to a set.
\\
\end{desctab}
For example:
\begin{showfile}
\verbatiminput{setfn.xt}
\end{showfile}
will produce:
\begin{showfile}
\verbatiminput{setfn.out}
\end{showfile}
\subsection{Regular expressions}
\label{s.regex}
The regular expressions in {\Tm} are based on the regular expressions
used in {\Unix} shells.  A pattern must match the entire string.
The following meta-characters are recognized:
\par
\begin{desctab}
\verb'?' & Matches any character.  \\
\verb'\'
&
Matches the character following it.
It is used as an escape character for all
meta-characters, including itself. When used
in a set (see below), it is treated as an ordinary character.
\\
{\tt [set]}
&
Matches one of the characters in the set.
If the first character in the set is `\verb!^!',
it matches a character {\em not} in the set. A shorthand like
\verb!a-z! is used to specify a set of
characters \verb!a! up to \verb!z!, inclusive. The special
characters `\verb!]!' and `\verb!-!' have no special
meaning if they appear as the first characters in the set.
\par
Examples:
\par
\begin{tabular}{ll}
\verb![a-z]! & Any lower case alpha. \\
\verb![^]-]! & Any char except `\verb!]!' and `\verb!-!'. \\
\verb![^A-Z]! & Any char except upper case alpha. \\
\verb![a-zA-Z]! & Any alpha.
\end{tabular}
\\
\verb!*!
&
Matches the longest possible span of zero or more arbitrary characters.
\\
\verb!(form)!
&
Matches what {\tt form} matches,
and assigns the matching pattern to one of the numbered patterns.
The patterns are numbered from left to right by their opening bracket.
\end{desctab}
\par
When patterns are substituted the substitution string can also have
some meta-characters:

\begin{desctab}
\par
\verb!&! & Is replaced by the entire matched pattern.  \\
\verb!\d!
&
Where \verb'd' is a digit,
is replaced by numbered pattern number `{\tt d}' as matched by
`{\tt (form)}'.
If the numbered pattern was not assigned in the original string,
it is empty. \verb!\0! is replaced by the entire matched pattern.
\\
\verb!\&! & Is replaced by `\verb!&!'.  \\
\verb!\\! & Is replaced by `\verb!\!'. \\
\end{desctab}
\par
\begin{desctab}
\\
{\tt filt ps pr e..e}\index{filt@{\tt filt}}
&
Given a search pattern {\tt ps} and a replacement pattern {\tt pr},
try to match all elements {\tt e} with {\tt ps},
and for each matching element return a copy of {\tt pr}
with the proper substitutions for any {\tt \&} and \verb!\d!.
\\
{\tt rmlist pat e..e}\index{rmlist@{\tt rmlist}}
&
Remove all elements matching {\tt pat} from the list.
\\
{\tt subs ps pr e..e}\index{subs@{\tt subs}}
&
Given a search pattern {\tt ps} and a replacement pattern {\tt pr},
copy all elements {\tt e},
and replace all elements matching {\tt ps} with {\tt pr}
with the proper substitutions for any {\tt \&} and \verb!\d!.
All elements that do not match {\tt ps} are copied without change.
\end{desctab}
For example:
\begin{showfile}
\verbatiminput{regex.xt}
\end{showfile}
will produce:
\begin{showfile}
\verbatiminput{regex.out}
\end{showfile}
\subsection{Environment access functions}
\index{environment access|(}
\nopagebreak
\par
\begin{desctab}
{\tt getenv nm}\index{getenv@{\tt getenv}}
&
Return the value of environment variable {\tt nm}.
If the variable does not exist an empty string is returned.
\\
{\tt isinenv nm}\index{isinenv@{\tt isinenv}}
&
Determine whether environment variable {\tt nm} exists.
Return {\it true} if it exists, or {\it false} \/if it does not.
\\
{\tt dsfilename}\index{dsfilename@{\tt dsfilename}}
&
Return the name of the file that describes the data structures.
\\
{\tt searchfile fn..fn}\index{searchfile@{\tt searchfile}}
&
Given a list of file names,
return a list
of full names files that have been located using the search path,
see the function {\verb+searchpath+}.
\par
If a file can not be found in the search path,
the string \verb+?+ (a single question mark) is returned for that
file name.
\\
{\tt searchpath}\index{searchpath@{\tt searchpath}}
&
Return the search path for \verb+.include+ and \verb+.insert+ files.
By default,
the search path contains 
the directory the variable \verb+libpath+ is set to.
If additional
directories are specified with the \verb+-I+ option,
they are appended to the list.
\\
{\tt tplfilename}\index{tplfilename@{\tt tplfilename}}
&
Return the name of the current template file.
\\
{\tt tpllineno}\index{tpllineno@{\tt tpllineno}}
&
Return the current line number in the current template file.
\\
{\tt matchvar pat}\index{matchvar@{\tt machvar}}
&
Given a regular expression as defined in section~\ref{s.regex},
return a list of all variables that match this regular expression.
In particular, \verb'${matchvar *}' will return a list of all
variables. The order of the returned list is arbitrary.
\\
{\tt matchmacro pat}\index{matchmacro@{\tt machmacro}}
&
Given a regular expression as defined in section~\ref{s.regex},
return a list of all macros that match this regular expression.
In particular, \verb'${matchmacro *}' will return a list of all
macros. The order of the returned list is arbitrary.
\\
{\tt defined v}\index{defined@{\tt defined}}
&
Return {\it true} if variable {\tt v} is defined,
or {\it false} otherwise.
\\
{\tt definedmacro m}\index{definedmacro@{\tt definedmacro}}
&
Return {\it true} if macro {\tt m} is defined,
or {\it false} otherwise.
\end{desctab}
\par
For example:
\begin{showfile}
\begin{verbatim}
${getenv SHELL}
\end{verbatim}
\end{showfile}
may produce (depending on the value of the \verb+SHELL+ environment variable):
\begin{showfile}
\begin{verbatim}
/bin/csh
\end{verbatim}
\end{showfile}
\index{environment access|)}
\subsection{Time functions}
\index{time functions|(}
\nopagebreak
\par
\begin{desctab}
{\tt now}\index{now@{\tt now}}
&
Return the current time as an integer number of seconds from the epoch
(system-dependent, but usually 1 jan 1970 00:00:00),
similar to the function \verb'time()' in the standard {\C} library.
\\
{\tt formattime time fmt}\index{formattime@{\tt formattime}}
&
Given a time in an integer number of seconds from the epoch, (as returned
by the function {\tt now}), return a formatted string for this time. The
format string has the same syntax as for the function \verb'strftime()'
in the {\C} library. That is, all characters are copied to the output
string, except for the following character sequences, which are replaced
by the listed string:
\begin{center}
\begin{tabular}{ll}
\verb'%a' & abbreviated weekday name \\
\verb'%A' & full weekday name \\
\verb'%b' & abbreviated month name \\
\verb'%B' & full month name \\
\verb'%c' & local date and time representation \\
\verb'%d' & day of the month \\
\verb'%H' & hour (24-hour clock) (00-23) \\
\verb'%I' & hour (12-hour clock) (01-12) \\
\verb'%j' & day of the year (001-366) \\
\verb'%m' & month (01-12) \\
\verb'%M' & minute (00-59) \\
\verb'%p' & local equivalent of AM or PM \\
\verb'%S' & second (00-61) \\
\verb'%U' & week number of the year (Sunday as 1st day of week) (00-53) \\
\verb'%w' & weekday (0-6, Sunday is 0) \\
\verb'%W' & week number of the year (Monday as 1st day of week) (00-53) \\
\verb'%x' & local date representation \\
\verb'%X' & local time representation \\
\verb'%y' & year without century (00-99) \\
\verb'%Y' & year with century \\
\verb'%Z' & time zone name, if any \\
\verb'%%' & \verb'%' \\
\end{tabular}
\end{center}
\par
By the default the format string \verb'%a %b %Y %H:%M:%S' is used.
\\
{\tt processortime}\index{processortime@{\tt processortime}}
&
Return the current amount of processor time that has been consumed by the
program as an integer number of milliseconds. Note that this is {\em
not} the elapsed time.
\par
Internally, the function \verb'clock()' from the standard {\C} library is
used. Although the returned time has a {\em resolution} of milliseconds,
it may have a much worse {\em accuracy}. The accuracy is system-dependent,
but typically there is an error of about 10ms.
\\
\end{desctab}
\par
For example, the input file:
\begin{showfile}
\verbatiminput{time.xt}
\end{showfile}
could produce (depending on time and processor speed) the following output:
\begin{showfile}
\verbatiminput{time.out}
\end{showfile}
\index{time functions|)}
\subsection{Data structure access functions}
\begin{desctab}
{\tt classlist}\index{classlist@{\tt classlist}}
&
Return the list of {\em class} types defined in the data structure file.
\\
{\tt conslist t..t}\index{conslist@{\tt conslist}}
&
Given a list of types \verb't', return the list of constructors of
each type.  If a type is not a constructor base type, and empty list
is returned for that type.
\\
{\tt ctypelist}\index{ctypelist@{\tt ctypelist}}
&
Return the list of {\em constructor} types defined in the data structure file.
\\
{\tt fields t..t}\index{fields@{\tt fields}}
&
Given a list of types {\tt t}, return the list of field names of each type.
Fields in superclasses of this type are {\em not} listed.
\\
{\tt allfields t..t}\index{allfields@{\tt allfields}}
&
Given a list of types {\tt t}, return the list of field names of each type,
including inherited fields.
The fields are ordered `height first', so that the fields of
superclasses are listed before the fields of their inheritors.
\\
{\tt isvirtual t}\index{isvirtual@{\tt isvirtual}}
&
Given a type {\tt t}, return {\tt 1} if the type is virtual, or
{\tt 0} if it is not. A type is virtual if it is a constructor base
type, if it is a class type containing a labeled component, or if it
is a class type defined as virtual (with the \verb'~=' operator).
\\
{\tt metatype t..t}\index{metatype@{\tt metatype}}
&
Given a list of types {\tt t}, return the metatype of each type.
If {\tt t} is defined in the
datastructure definition file, return one of the strings {\tt class},
{\tt tuple}, {\tt constructor}, {\tt constructorbase}, or {\tt alias}.
If {\tt t} has the prefix {\tt listpre} and the suffix {\tt listsuff},
return the string {\tt list}; in all other cases return the string {\tt atom}.
\\
{\tt tuplelist}\index{tuplelist@{\tt tuplelist}}
&
Return the list of {\em tuple} types defined in the data structure file.
\\
{\tt type t e}\index{type@{\tt type}}
&
Given a type {\tt t} and a field name {\tt e}, construct the proper
name of this type from the type name and list level of this field.
The list type name is constructed by prefixing the type name with the
correct number of copies of {\tt listpre}, and suffixing the
type name with the correct number of copies of {\tt listsuff}.
This function is equivalent with the expression
\verb'${mklist ${typelevel $t $e} ${typename $t $e}}'.
\par
Field {\tt e} may belong to the given type or one of the superclasses
of the type.
\\
{\tt types t..t}\index{types@{\tt types}}
&
Given a list of types, return the proper type names of the fields of
these types. The types of inherited fields are {\em not} returned.
The proper type names are constructed as described for the function
{\tt type}.
The types are returned in such an order that they match up with the
field names in the order returned by {\tt fields}.
\\
{\tt alltypes t..t}\index{alltypes@{\tt alltypes}}
&
Given a list of types, return the proper type names of the fields of
these types, including those of inherited fields.
The proper type names are constructed as described for the function
{\tt type}.
The types are returned in such an order that they match up with the
field names in the order returned by {\tt allfields}.
\\
{\tt typelevel t e}\index{typelevel@{\tt typelevel}}
&
Given a type {\tt t} and a field name {\tt e}, return return the level
of list bracketing of the field. Thus, a field of type \verb't' has list
level 0, a field of type \verb'[t]' has list level 1, and so on.
\par
Field {\tt e} may belong to the given type or one of the superclasses
of the type.
\\
{\tt typelist}\index{typelist@{\tt typelist}}
&
Return the list of types defined in the data structure file.
\\
{\tt typename t e}\index{typename@{\tt typename}}
&
Given a type {\tt t} and a field name {\tt e}, return the type of that field.
\par
Field {\tt e} may belong to the given type, or to one of the superclasses
of the type.
\\
\end{desctab}
For example, given the data structure definitions:
\begin{showfile}
\verbatiminput{fnds.xt}
\end{showfile}
and the template:
\begin{showfile}
\verbatiminput{fnds.xt}
\end{showfile}
the following output is produced:
\begin{showfile}
\verbatiminput{fnds.out}
\end{showfile}
\subsection{Alias functions}
These functions are used to access alias types.
\begin{desctab}
{\tt aliases}\index{aliases@{\tt aliases}}
&
Return the list of alias types defined in the data structure file.
\\
{\tt alias t..t}\index{alias@{\tt alias}}
&
Apply the alias definitions to the given types. For each type, if no alias
type is defined, return the type itself, else return the alias of the type.
\\
\end{desctab}
For example, given the data structure definitions shown on page
\pageref{plotds} and the following input file:
\begin{showfile}
\verbatiminput{aliasfn.xt}
\end{showfile}
the following output is produced:
\begin{showfile}
\verbatiminput{aliasfn.out}
\end{showfile}
\subsection{Class inquiry functions}
The functions listed below all access the inheritance information
of the defined types.
\begin{desctab}
{\tt inherits t..t}\index{inherits@{\tt inherits}}
&
Given a list of types, return the list of types this type inherits directly.
\par
If a type is inherited by several types, it will be listed several times.
\\
{\tt inheritors t..t}\index{inheritors@{\tt inheritors}}
&
Given a list of types,
return the list of types that inherit directly from this type.
\\
{\tt subclasses t..t}\index{subclasses@{\tt subclasses}}
&
Given a list of types,
return the list of types that inherit directly or indirectly from these types.
\\
{\tt superclasses t..t}\index{superclasses@{\tt superclasses}}
&
Given a list of types, return the list of types this type inherits directly
or indirectly.
\par
If a type is inherited by several types, it will be listed several times.
\\
\end{desctab}
For example:
\begin{showfile}
\verbatiminput{fnclass.xt}
\end{showfile}
will produce (assuming the data structure definitions of page \pageref{plotds}):
\begin{showfile}
\verbatiminput{fnclass.out}
\end{showfile}
\subsection{Code generation service functions}
These functions implement some of the complicated operations that
are necessary during code generation. 
\par
One definition must be introduced:
A type $t$ {\em depends} on a type $s$ if $s$
is used in at least one of the elements of $t$,
or if $t$ has a type in one of its elements that depends on $s$.
Also,
a list of a type depends on that type.
Primitive types depend on nothing.
\par
For example:
in the data structure on page \pageref{plotds} type {\tt plot}
depends on the single types {\tt xypoint}, {\tt int}, and {\tt num},
and on the list types {\tt [xypoint]} and {\tt [num]}.
\par
Several functions in this group refer to the {\Tm} variables \verb+listpre+
and \verb+listsuff+ for a prefix and a suffix respectively to construct
a list name from a single type name.
If one of these variables is not defined, an empty string is assumed.
\par
\begin{desctab}
{\tt depsort t..t}\index{depsort@{\tt depsort}}
&
Given a list of types,
rearrange them so that all types that contain a certain type
are placed after the type itself.
Thus, if a type {\tt ta} uses a type {\tt tb}, {\tt tb} will
be placed before {\tt ta};
this is in effect a topological sort.
The function will complain about circularities.
\par
List types are considered to contain their element type.
\par
The function does not consider inheritance relations.
\\
{\tt inheritsort t..t}\index{inheritsort@{\tt inheritsort}}
&
Given a list of types,
rearrange them so that all types that all types are placed after
the types they inherit from.
\\
{\tt delisttypes t..t}\index{delisttypes@{\tt delisttypes}}
&
Given a list of types, return a list of the element types of the list types
by removing the list prefix {\tt listpre} the list suffix {\tt listsuff}.
For types that don't have this prefix and suffix nothing is returned.
\\
{\tt listtypes t..t}\index{listtypes@{\tt listtypes}}
&
Given a list of types, return a list of all types that are prefixed
by {\tt listpre} and suffixed by {\tt listsuff}.
\\
{\tt mklist n e..e}\index{mklist@{\tt mklist}}
&
Given a list level {\tt n} and a list of type names {\tt e}, construct
for each type a list name for the given list level by prefixing each
type name with {\tt n} copies of \verb+listpre+ and suffixing each type
name with {\tt n} copies of \verb+listsuff+.
\\
{\tt singletypes t..t}\index{singletypes@{\tt singletypes}}
&
Given a list of types, return a list of all types that are not prefixed
by {\tt listpre} or are not suffixed by {\tt listsuff}.
\\
{\tt stemname t..t}\index{stemname@{\tt stemname}}
&
Given a list of list type names {\tt t},
return a list of stem names.
All pairs of list prefixes and suffices are stripped from all
the type names.
The list prefix is taken from the variable \verb+listpre+,
the list suffix is taken from the variable \verb+listsuff+.
If both are empty the list is returned unchanged, and an error
is emitted.
\\
{\tt virtual t..t}\index{virtual@{\tt virtual}}
&
Given a list of types {\tt t}, return all types that are virtual.
\\
{\tt nonvirtual t..t}\index{nonvirtual@{\tt nonvirtual}}
&
Given a list of types {\tt t}, return all types that are not virtual.
\\
{\tt reach t..t "" b..b}\index{reach@{\tt reach}}
&
Given a list of types {\tt t}, and a list of blocking types {\tt b},
return all types that can be reached from
the types {\tt t} without visiting any of the types {\tt b}.
The list of starting types and the list of blocking types is separated
by an empty string ({\tt ""}). If there is no empty string, there are
no blocking types.
\par
The \defn{direct reach} of a class, tuple, constructor
or constructor-base type are the types of its fields and inheritors;
the direct reach of a list type is its element type, the direct reach
of an alias type is the target type of the alias, and the direct reach
of an atomic type is empty.
\par
The \defn{reach} of a type is the union of the type itself, the direct
reach of the type, and the reach of the types in the direct reach of
the type.  In other words, the reach is the transitive closure of the
direct reach.
\\
\end{desctab}
For example:
\begin{showfile}
\verbatiminput{service.xt}
\end{showfile}
will produce (assuming the data structure definitions of page \pageref{plotds}):
\begin{showfile}
\verbatiminput{service.out}
\end{showfile}
\subsection{Obsolete data structure access functions}
\par
These functions have been used in previous version of {\Tm} to retrieve
information about the data structures.
They have been replaced by the functions in the previous sections,
and should not be used in new templates.
\begin{desctab}
{\tt ttypeclass t e}\index{ttypeclass@{\tt ttypeclass}}
&
Given a type {\tt t} and a field name {\tt e}, return the type class
of that field.  Possible type classes are {\tt single} and {\tt list}
for a single element and a list of elements respectively.
\par
Field {\tt e} may belong to the given type or one of the superclasses
of the type.
\par
This function is obsolete; use \verb'typelevel' instead.
\\
{\tt ttypelist}\index{ttypelist@{\tt ttypelist}}
&
Equivalent to {\tt tuplelist}.
\\
{\tt telmlist t}\index{telmlist@{\tt telmlist}}
&
Equivalent to {\tt field}.
\\
{\tt ttypeclass t e}\index{ttypeclass@{\tt ttypeclass}}
&
Equivalent to {\tt typeclass}.
\\
{\tt ttypellev t e}\index{ttypellev@{\tt ttypellev}}
&
Equivalent to {\tt typelevel}.
\\
{\tt ttypename t e}\index{ttypename@{\tt ttypename}}
&
Equivalent to {\tt typename}.
\\
{\tt celmlist t c [inherited]}\index{celmlist@{\tt celmlist}}
&
Given a constructor type {\tt t}, and a constructor name {\tt c},
return the list of element names of that constructor.
If {\tt t} is a tuple type, an error message is given.
\par
If a third parameter is given, and this parameter is the string
{\tt inherited}, the fields of the superclasses of this constructor are also
listed. The fields are ordered `height first', so that the fields of
superclasses are listed before the fields of the types that inherit these
superclasses.
\par
This function is obsolete; use \verb'${fields c}' instead.
\\
{\tt ctypeclass t c e}\index{ctypeclass@{\tt ctypeclass}}
&
Given a constructor type {\tt t}, a constructor {\tt c} and a constructor
element name {\tt e}, return the type class of that element.  Possible
type classes are {\tt single} and {\tt list} for a single element and
a list of elements respectively.  If {\tt t} is a tuple type, an error
message is given.
\par
This function is obsolete; use \verb'${typeclass c}' instead.
\\
{\tt ctypellev t c e}\index{ctypellev@{\tt ctypellev}}
&
Given a type {\tt t}, a constructor {\tt c} and a constructor element name
{\tt e}, return return the level of list bracketing of the element. Thus,
an element of type \verb't' has list level 0, an element of type
\verb'[t]' has list level 1, and so on.  If {\tt t} is a tuple type,
an error message is given.
\par
This function is obsolete; use \verb'${typelevel c}' instead.
\\
{\tt ctypename t c e}\index{ctypename@{\tt ctypename}}
&
Given a type {\tt t},
a constructor {\tt c}
and a constructor element name {\tt e},
return the type of that element.
If {\tt t} is not a constructor base type, an error message is given.
\par
This function is obsolete; use \verb'${typename c}' instead.
\\
\end{desctab}
\subsection{Deferred evaluation}
\begin{desctab}
{\tt call m p1..pn}\index{call@{\tt call}}
&
Given a macro name \verb+m+ and a list of parameters \verb+p1..pn+,
invoke macro with the given list of parameters.
Before execution,
the formal parameters of the macro are set to the values in the parameter list.
To pass a list of words to a formal parameter,
it must be surrounded by double quotes.
The number of parameters must match the number of formal parameters,
or else an error message is given.
Note that each parameter word corresponds to one formal parameter.
Macros `see' the variable values and macros that are in effect at the moment
of execution of the macro.
All changes that are made to variables or macros within a macro
are invisible outside the macro.
During execution of the macro at least one \verb+.return+ command must be
given to record a return value for the macro.
The macro is not allowed to generate output, and therefore may only contain
line commands.
If you want to generate output, 
use the \verb+.call <macro> <parm>..<parm>+ form instead.
\\
{\tt eval e1..en}\index{eval@{\tt eval}}
&
Given a list of expressions \verb+e1..en+, evaluate all expressions,
and return them in a list.
\\
\end{desctab}
For example:
\begin{showfile}
\verbatiminput{eval.xt}
\end{showfile}
will produce:
\begin{showfile}
\verbatiminput{eval.out}
\end{showfile}
\section{Arithmetic expressions}
\label{s.intexpr}
\index{arithmetic expressions|(}
Arithmetic expressions have the form
\begin{showfile}
\begin{verbatim}
$[<expr>]
\end{verbatim}
\end{showfile}
Each time the characters `{\tt \$[}' are encountered,
the string up to the next unbalanced `{\tt ]}' is evaluated,
and the resulting string is
interpreted as an integer expression.
The following operators are available, where the number in the first column
indicates priority; the lower the number, the stronger an operator binds:
\begin{quote}
\begin{tabular}{lll}
priority & symbol & function \\
0 & \verb'()' & Priority brackets. \\
1 & \verb'-' & Unary minus. \\
1 & \verb'!' & Boolean not. \\
1 & \verb'+' & Unary plus. \\
2 & \verb'*' & Multiplication. \\
2 & \verb'/' & Division. \\
2 & \verb'%' & Modulus. \\
3 & \verb'+' & Addition. \\
3 & \verb'-' & Subtraction.\\
4 & \verb'!=' & Not equal to. \\
4 & \verb'==' & Equal to. \\
4 & \verb'<=' & Less or equal. \\
4 & \verb'<' & Less. \\
4 & \verb'>=' & Greater or equal. \\
4 & \verb'>' & Greater. \\
5 & \verb'&' & Boolean and. \\
5 & \verb'|' & Boolean or. \\
\end{tabular}
\end{quote}
The integer `0' represents `false',
all other integers represent `true'.
Boolean operators always return `true' as `1'.
Binary operators are right binding.
That is,
\[ a\ {\rm op}\ b\ {\rm op}\ c = a\ {\rm op}\ (b\ {\rm op}\ c) \]
For example:
\begin{showfile}
\verbatiminput{math.xt}
\end{showfile}
will produce:
\begin{showfile}
\verbatiminput{math.out}
\end{showfile}
\index{arithmetic expressions|)}
